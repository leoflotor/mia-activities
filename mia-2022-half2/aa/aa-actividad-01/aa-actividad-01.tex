% -- Document configuration
\documentclass{article}

% -- Input and language settings
% \usepackage[utf8]{inputenc}
\usepackage[spanish]{babel}
\decimalpoint                             % From bable package to use points instead of commas in decimals

% -- Page and line settings
\usepackage{geometry}
\geometry{letterpaper, 
    % margin=2cm, 
    left=2cm, right=2cm,
    top=1.2cm, bottom=1.2cm,
    includefoot, 
    includehead}
\renewcommand{\baselinestretch}{1.2}

% -- Required packages
\usepackage{xcolor}
\usepackage[many]{tcolorbox}
\usepackage{blkarray}                       % Allow matrices with row and column labels
\usepackage{mathtools,amsfonts,amsmath}     % Loads amsmath if not already loaded
\allowdisplaybreaks                         % To allow page breaks if equations are too long
\usepackage[parfill]{parskip}               % No indent and separation lines for paragraphs
\usepackage{cancel}                         % To cancel math terms
\usepackage[shortlabels]{enumitem}          % To handle enumerations
\usepackage{tikz}
\usetikzlibrary{plotmarks}
\usetikzlibrary{math}
\usepackage{pgfplots}
\pgfplotsset{compat=newest}
\usepackage[mode=buildnew]{standalone}      % To import figures in standalone files
\usepackage[hidelinks]{hyperref}
\usepackage[spanish]{cleveref}              % To use autocompleted reference labels, language must be change as in babel package
\usepackage{caption}                        % Caption and subcaption to allow subfigures
\usepackage{subcaption}
\usepackage{float}                          % To specify the location of figures
\usepackage{multicol}                       % To use multicolumns
\usepackage[bottom]{footmisc}               % To locate footnotes at the bottom

% -- Title and heading settings
\usepackage{titling}
\usepackage{fancyhdr}
\pagestyle{fancy}

% -- Code and code formatting
\usepackage{minted}                         % To insert code
\usemintedstyle[julia]{gruvbox-light}       % Code theme and language
\definecolor{bg}{rgb}{0.98, 0.97, 0.88}     % Code block background

\usepackage{textcomp}                       % To allow straight quotes
\usepackage{fontspec}                       % To allow the use of monospace fonts
\setmonofont{JuliaMono}[Path=./codefonts/, Extension=.ttf, UprightFont=*-Regular, ItalicFont=*-RegularItalic, Scale=0.75]

\usepackage{fancyvrb}                       % To change line number font
\renewcommand{\theFancyVerbLine}{\textcolor{gray}{\footnotesize\texttt{\arabic{FancyVerbLine}}}}

\definecolor{light-gray}{gray}{0.95}        % Color, box and style to show small code thingys inside normal text
\newcommand{\code}[1]{\colorbox{light-gray}{\texttt{#1}}}

% -- Bilbiography preferences
\usepackage[square,numbers]{natbib}
\bibliographystyle{unsrt}

% -- Footnotes without numbering
\newcommand\nnfootnote[1]{%
  \begin{NoHyper}
  \renewcommand\thefootnote{}\footnote{#1}%
  \addtocounter{footnote}{-1}%
  \end{NoHyper}
}

% -- Theorems
\newtheorem{theorem}{Theorem}

\lhead{\theinstitution\ -- \thedepartment}
\chead{}
\rhead{\thecourse\ -- \thetitle}
\lfoot{}
\cfoot{\thepage}
\rfoot{}

% -- Problem solution
\newenvironment{solution}
{\begin{quote}
\textbf{Solución:}\medskip

}
{

% \hfill\rule{0.5\textwidth}{0.5pt}
\end{quote}}

% -- Equation result
\newcommand{\result}[1]
{
\tcbhighmath[colframe=white, colback=gray!15, sharp corners]
{#1}
}

% -- Function definitions
\newcommand{\dprod}[2]{{#1} \cdot {#2}}
\newcommand{\txtgray}[1]{\textcolor{gray}{#1}}
\newcommand{\quotes}[1]{\textquotedbl#1\textquotedbl}

% -- Author information
\title{Actividad 6}
\author{Leonardo Flores Torres}
\newcommand\theinstitution{Universidad Veracruzana}
\newcommand\thedepartment{Inteligencia Artificial}
\newcommand\thecourse{Análisis de Algoritmos}

% -- Paths
\newcommand\codepath{../MiniFinder/src/MiniFinder.jl}


% -- Document
\begin{document}

\thispagestyle{empty}

%Title
\begin{center}
\textsc{\theinstitution}\\[2mm]

\thedepartment

\rule{0.6\textwidth}{0.5pt}\\[2mm]

\thecourse \\[4mm]

{\Large \textbf{\thetitle}}\\[2mm]

\theauthor \\[2mm]

{\small \today}
\end{center}
\medskip

Resolver las siguientes relaciones de recurrencia.
\begin{enumerate}[label=\textbf{\alph*.}]
    % -- Primer problema
    \item $x(n) = x(n-1) + 5$ para $n > 1$, $x(1) = 0$.
    \begin{solution}
        Primero se utilizó \textit{forward substitution} para trarar de encontrar un patrón que se pueda identificar fácilmente, y se encontró que se podía escribir de la siguiente forma,
        \begin{equation*}
            \begin{rcases}
                \begin{array}{l l l}
                    \txtgray{n} & x(n) & \\
                    \txtgray{1} & x(1) = 0 & = \dprod{0}{5} = \dprod{(1 - 1)}{5} \\
                    \txtgray{2} & x(2) = x(1) + 5 = 5 & = \dprod{1}{5} = \dprod{(2 - 1)}{5} \\
                    \txtgray{3} & x(3) = x(2) + 5 = 10 & = \dprod{2}{5} = \dprod{(3 - 1)}{5} \\
                    \txtgray{4} & x(4) = x(3) + 5 = 15 & = \dprod{3}{5} = \dprod{(4 - 1)}{5}
                \end{array} \quad
            \end{rcases}
            x(k) = \dprod{(k - 1)}{5},\ k \geq 1\ .
        \end{equation*}

        Posteriormente, se aplicó \textit{backward substitution} para encontrar una relación de recurrencia,
        \begin{equation*}
            \begin{rcases}
                \begin{array}{l l l l}
                    \txtgray{n} \quad & x(n) & = x(n - 1) + 5 & \\
                    \txtgray{n - 1} \quad & & = \left[ x(n - 2) + 5 \right] + 5 & = x(n-2) + \dprod{2}{5} \\
                    \txtgray{n - 2} \quad & & = \left[ x(n - 3) + 5 \right] + \dprod{2}{5} & = x(n - 3) + \dprod{3}{5} \\
                    \txtgray{n - 3} \quad & & = \left[ x(n - 4) + 5 \right] + \dprod{3}{5} & = x(n - 4) + \dprod{4}{5} \\
                    \quad\txtgray\vdots & & \qquad\vdots & \qquad\vdots
                \end{array} \quad
            \end{rcases}
            x_{n - i + 1} = x(n - i) + \dprod{i}{5},\ i \geq 1\ ,
        \end{equation*}
        de donde se encontró que el valor para el cual se llegaba al final de $x(n)$ está dado por,
        \begin{equation*}
            \begin{rcases}
                \begin{array}{l}
                    x(1) = 0 \\
                    x(1) = x(n - i)
                \end{array}
            \end{rcases}
            \begin{array}{l}
                n - i = 1 \\
                i = n - 1
            \end{array}
            \ ,
        \end{equation*}
        donde $1 \leq i \leq n - 1$. Al usar la relación de recurrencia se obtiene que
        \begin{equation*}
            \begin{array}{l l l l}
                \txtgray\vdots & & \qquad\vdots & \qquad\vdots \\
                \txtgray{2} \quad & x(n) & = x\left(n - (n - 1)\right) + \dprod{(n - 1)}{5} & = \cancelto{0}{x(1)} + \dprod{(n - 1)}{5}\ .
            \end{array}
        \end{equation*}
        
        Finalmente, el resultado es
        \begin{equation}
            \result{x(n) = \dprod{(n - 1)}{5}\ .}
        \end{equation}

        Este procedimiento es el mismo que se repetirá para el resto de los problemas.
    \end{solution}

    % -- Segundo problema
    \item $x(n) = 3 x(n-1)$ para $n > 1$, $x(1) = 4$.
    \begin{solution}
        La solución encontrada usando el método de \textit{forward substitution} es
        \begin{equation*}
            \begin{rcases}
                \begin{array}{l l l}
                    \txtgray{n} & x(n) & \\
                    \txtgray{1} & x(1) = 4 & = \dprod{4}{3^0} \\
                    \txtgray{2} & x(2) = \dprod{3}{x(1)} & = \dprod{4}{3^1} \\
                    \txtgray{3} & x(3) = \dprod{3}{x(2)} & = \dprod{4}{3^2} \\
                    \txtgray{4} & x(4) = \dprod{4}{x(3)} & = \dprod{4}{3^3}
                \end{array} \quad
            \end{rcases}
            x(k) = \dprod{4}{3^{k - 1}},\ k \geq 0\ .
        \end{equation*}
        De manera similar al problema anterior, por el método de \textit{backward substitution}, se encuentra que
        \begin{equation*}
            \begin{rcases}
                \begin{array}{l l l l}
                    \txtgray{n} & x(n) & = \dprod{3}{x(n - 1)} & = \dprod{3^1}{x(n - 1)} \\
                    \txtgray{n - 1} & & = 3 \left[ \dprod{3}{x(n - 2)} \right] & = \dprod{3^2}{x(n - 2)} \\
                    \txtgray{n - 2} & & = 3 \left[ \dprod{3}{x(n - 3)} \right] & = \dprod{3^3}{x(n - 3)} \\
                    \txtgray{n - 3} & & = 3 \left[ \dprod{3}{x(n - 4)} \right] & = \dprod{3^4}{x(n - 4)} \\
                    \quad\txtgray\vdots & & \qquad \vdots & \qquad \vdots
                \end{array} \quad
            \end{rcases}
            x_{n - i + 1} = \dprod{3^i}{x(n - i)}\ , i \geq 1\ .
        \end{equation*}
        Además, se puede utilizar la condición inicial $x(1)$ para determinar el último valor de la sucesión,
        \begin{equation*}
            \begin{rcases}
                \begin{array}{l}
                    x(1) = 4 \\
                    x(1) = x(n - i)
                \end{array}
            \end{rcases}
            \begin{array}{l}
                n - i = 1 \\
                i = n - 1
            \end{array}\ ,
        \end{equation*}
        donde $1 \leq i \leq n - 1$, y el último valor de la sucesión está dado por
        \begin{equation*}
            \begin{array}{l l l l}
                \txtgray\vdots & & \qquad\vdots & \qquad\vdots \\
                \txtgray{2} & x(n) & = \dprod{3^{n - 1}}{x\left(n - (n - 1)\right)} & = \dprod{3^{n - 1}}{\cancelto{4}{x(1)}}\ .
            \end{array}
        \end{equation*}
        El resultado final es,
        \begin{equation}
            \result{x(n) = \dprod{4}{3^{n - 1}}\ .}
        \end{equation}
    \end{solution}

    % -- Tercer problema
    \item $x(n) = x(n-1) + n$ para $n > 0$, $x(0) = 0$.
    \begin{solution}
        Se usó el método de \textit{forward substitution} pero no se logró encontrar un patrón para determinar la solución.
        \begin{equation*}
            \begin{rcases}
                \begin{array}{l l l}
                    \txtgray{n} & x(n) & \\
                    \txtgray{0} & x(0) & = 0 \\
                    \txtgray{1} & x(1) = x(0) + 1 & = 1 \\
                    \txtgray{2} & x(2) = x(1) + 2 & = 3 \\
                    \txtgray{3} & x(3) = x(2) + 3 & = 6
                \end{array} \quad
            \end{rcases}
            \text{¡Patrón no tan evidente!}
        \end{equation*}

        Debido a lo anteriormente mencionado se dió espacial atención al caso usando \textit{backward substitution} para encontrar la relación de recurrencia, 
        \begin{equation*}
            \underbrace{%
            \begin{array}{l l l l}
                \txtgray{n} & x(n) & = x(n - 1) + n & \\
                \txtgray{n - 1} & & = \left[x(n - 2) + (n - 1)\right] + n & = x(n - 2) + \sum_{i=0}^{1} (n - i) \\
                \txtgray{n - 2} & & = \left[x(n - 3) + (n - 2)\right] + \sum_{i=0}^{1} (n - i) & = x(n - 3) + \sum_{i=0}^{2} (n - i) \\
                \txtgray{n - 3} & & = \left[x(n - 4) + (n - 3)\right] + \sum_{i=0}^{2} (n - i) & = x(n - 4) + \sum_{i=0}^{3} (n - i) \\
                \quad\txtgray\vdots & & \qquad \vdots & \qquad \vdots
            \end{array}}_{%
            \begin{array}{c}
                x_{n - i} = x(n - 1 - i) + \sum_{k=0}^{i} (n - k),\ i \geq 0 \ .
            \end{array}}
        \end{equation*}
        Usando la condición inicial, y comparando con la relación de recurrencia, se encontró el último valor de la sucesión
        \begin{equation*}
            \begin{rcases}
                \begin{array}{l}
                    x(0) = 0 \\
                    x(0) = x(n - 1 - i)
                \end{array}
            \end{rcases}
            \begin{array}{l}
                n - 1 - i = 0 \\
                i = n - 1
            \end{array}\ ,
        \end{equation*}
        donde $0 \leq i \leq n - 1$. Y la sucesión termina en
        \begin{equation*}
            \begin{array}{l l l l}
                \txtgray\vdots & & \qquad \vdots & \qquad \vdots \\
                \txtgray{1} & x(n) & = x\left(n - 1 (n - 1)\right) + \sum_{k=0}^{n-1} (n - k) & = \cancelto{0}{x(0)} + \sum_{k=0}^{n-1} (n - k)\ .
            \end{array}
        \end{equation*}

        A pesar de haber llegado ya a un resultado no parece ser que pudiera considerarse la solución al problema ya que desde un inicio en el caso de \textit{forward substitution} el comportamiento indicaba una sumataoria de términos. Se decidió tomar dicha sumatoria y buscar una manera de reducirla como se muestra a continuación,
        \begin{equation*}
            \begin{array}{l l}
                x(n) & = \sum_{k=0}^{n-1} (n - k) \\
                & = n + (n - 1) + (n - 2) + \ldots + \cancelto{3}{\left[n - (n - 3)\right]} + \cancelto{2}{\left[n - (n - 2)\right]} + \cancelto{1}{\left[n - (n - 1)\right]} \\
                & = 1 + 2 + 3 + \ldots + (n - 2) + (n - 1) + n \\
                & = \sum_{k=1}^{n} k\ .
            \end{array}
        \end{equation*}
        
        Sumando ambas sumas y resolviendo para $x(n)$,
        \begin{equation*}
            \begin{array}{l l}
                x(n) + x(n) & = \sum_{k=0}^{n-1} (n - k) + \sum_{k=1}^{n} k \\
                \dprod{2}{x(n)} & = \sum_{k=1}^{n} (n - k - 1) + \sum_{k=1}^{n} k \\
                & = \sum_{k=1}^{n} (n + 1) \\
                & = n (n + 1) \\
                x(n) & = \frac{n}{2} (n + 1)\ .
            \end{array}
        \end{equation*}

        Por lo tanto, el resultado al problema es
        \begin{equation}
            \result{x(n) = \frac{n}{2} (n + 1)\ .}
        \end{equation}

        Es curioso que la suma de 1 hasta $n$, donde $n \in \mathbb{N}$, se pueda expresar de esta manera tan elegante.

    \end{solution}
    
    % -- Cuarto problema
    \item $x(n) = x(n/2) + n$ para $n > 1$, $x(1) = 1$ (resolver para $n = 2^k$).
    \begin{solution}
        Primeramente se realizó el cambio en $x(n)$ de $n \to 2^k$,
        \begin{equation*}
            x(2^k) = x(2^{k - 1}) + 2^k\ .
        \end{equation*}
        Después se calcularon varios casos usando \textit{forward substitution}. El patrón encontrado no es tan evidente de detectar si se observa solamente el valor de $k$, pero al comparar el resultado de cada intento con su respectivo valor de $n$ puede resultar mas sencillo como se muestra a continuación.
        \begin{equation*}
            \begin{rcases}
                \begin{array}{l l l l l}
                    \txtgray{n} & \txtgray{k} & x(2^k) & & \\
                    \txtgray{1} & \txtgray{0} & x(2^0) & = 1 & = 1 \\
                    \txtgray{2} & \txtgray{1} & x(2^1) = x(2^0) + 2^1 & = 1 + 2^1 & = 3 \\
                    \txtgray{4} & \txtgray{2} & x(2^2) = x(2^1) + 2^2 & = 1 + 2^1 + 2^2 & = 7 \\
                    \txtgray{8} & \txtgray{3} & x(2^3) = x(2^2) + 2^3 & = 1 + 2^1 + 2^2 + 2^3 & = 15
                \end{array} \quad
            \end{rcases}
            x(n) = \dprod{2}{n} - 1 = 2^{k + 1} - 1\ .
        \end{equation*}
        La relación de recurrencia se encuentra usando \textit{backward substitution},
        \begin{equation*}
            \underbrace{%
            \begin{array}{l l l l}
                \txtgray{k} & x(2^k) & = x(2^{k - 1}) + 2^k & \\
                \txtgray{k - 1} & & = \left[x(2^{k - 2}) + 2^{k - 1}\right] + 2^k & = x(2^{k - 2}) + \sum_{i=0}^{1} (2^{k - j}) \\
                \txtgray{k - 2} & & = \left[x(2^{k - 3}) + 2^{k - 2}\right] + \sum_{i=0}^{1} (2^{k - j}) & = x(2^{k - 3}) + \sum_{i=0}^{2} (2^{k - j}) \\
                \txtgray{k - 3} & & = \left[x(2^{k-4}) + 2^{k - 3}\right] + \sum_{i=0}^{2} (2^{k - j}) & = x(2^{k - 4}) + \sum_{i=0}^{3} (2^{k - j}) \\
                \quad\txtgray\vdots & & \qquad \vdots & \qquad \vdots
            \end{array}}_{%
            \begin{array}{c}
                x_{k - i + 1} = x(2^{k - i}) + \sum_{j=0}^{i-1} 2^{k - j}\ .
            \end{array}}
        \end{equation*}
        De la condición inicial y de la relación de recurrencia se puede encontrar que
        \begin{equation*}
            \begin{rcases}
                \begin{array}{l l l}
                    x(1) & = x(2^0) & = 1 \\
                    x(2^0) & = x(2^{k - i})
                \end{array}
            \end{rcases}
            \begin{array}{l l}
                2^0 = 2^{k - i} \\
                0 = k - i
            \end{array}
            \implies i = k\ .
        \end{equation*}
        Sustituyendo $i = k$ en la relación de recurrencia se encuentra el último término de $x(n)$,
        \begin{equation*}
            \begin{array}{l l l l}
                \txtgray\vdots & & \qquad \vdots & \qquad \vdots \\
                \txtgray{1} & x(n) & = x(2^{k - k}) + \sum_{j=0}^{k-1} 2^{k - j} & = \underbrace{x(2^0)}_{1=2^{k - k}} + \sum_{j=0}^{k-1} 2^{k - j} \\
                & x(n) & = \sum_{j=0}^{k} 2^{k - j}\ ,
            \end{array}
        \end{equation*}
        donde $0 \leq j \leq k$. Esta suma se puede simplificar,
        \begin{equation*}
            \begin{array}{l l}
                x(2^k) & = \sum_{j=0}^{k} 2^{k - j} \\
                & = 2^{k} + 2^{k - 1} + 2^{k - 2} + \ldots + \overbrace{2^{k - (k - 2)}}^{2^2} + \overbrace{2^{k - (k - 1)}}^{2^1} + \overbrace{2^{k - k}}^{2^0} \\
                & = \sum_{j = 0}^{k} 2^j\ .
            \end{array}
        \end{equation*}
        Finalmente, haciendo un poco de manipulación a la serie se llega al r esultado que se encontró por el método de \textit{forward substitution},
        \begin{equation*}
            \begin{array}{l l}
                x(2^k) & = \sum_{j = 0}^{k} 2^j \\
                \dprod{2^1}{x(2^k)} & = \sum_{j = 1}^{k+1} 2^{j} \\
                \dprod{2^1}{x(2^k)} & = \sum_{j = 1}^{k} 2^{j} + 2^{k+1} \\
                & = \sum_{j = 1}^{k} 2^{j} + 2^{k+1} + (2^0 - 2^0) \\
                & = \sum_{j = 0}^{k} 2^{j} + 2^{k+1} - \underbrace{2^0}_{=1} \\
                \dprod{2^1}{x(2^k)} & = x(2^k) + 2^{k+1} - 1 \\
                x(2^k) & = 2^{k+1} - 1\ .
            \end{array}
        \end{equation*}
        La serie anterior es conocida como una \textit{serie geométrica}, y el procedimiento para resolverla es el comunmente encontrado en la litertura. El resultado a este problema es,
        \begin{equation}
            \result{x(2^k) = 2^{k+1} - 1\ .}
        \end{equation}
    \end{solution}

    % -- Quinto problema
    \item $x(n) = x(n/3) + 1$ para $n > 1$, $x(1) = 1$ (resolver para $n = 3^k$).
    \begin{solution}
        Nuevamente se hizo el cambio en $x(n)$ de $n \to 3^k$,
        \begin{equation*}
            x(3^k) = x(3^{k - 1}) + 1\ .
        \end{equation*}
        En este problema fue sencillo de identificar la dependencia de $k$ al usar \textit{forward substitution},
        \begin{equation*}
            \begin{rcases}
                \begin{array}{l l l l}
                    \txtgray{n} & \txtgray{k} & x(3^k) & \\
                    \txtgray{1} & \txtgray{0} & x(3^0) = x(1) & = 1 \\
                    \txtgray{3^1} & \txtgray{1} & x(3^1) = x(3^0) + 1 & = 2 \\
                    \txtgray{3^2} & \txtgray{2} & x(3^2) = x(3^1) + 1 & = 3 \\
                    \txtgray{3^3} & \txtgray{3} & x(3^3) = x(3^2) + 1 & = 4
                \end{array} \quad
            \end{rcases}
            x(3^k) = k + 1 = \log_{3}(n) + 1\ .
        \end{equation*}
        Por completez se repitió lo mismo que para los problemas anteriores, se calcularon varios términos usando \textit{backward substitution} y se dedujo la relación de recurrencia,
        \begin{equation*}
            \begin{rcases}
                \begin{array}{l l l}
                    \txtgray{k} & x(3^k) & = x(3^{k - 1}) + 1 \\
                    \txtgray{k - 1} & & = x(3^{k - 2}) + 2 \\
                    \txtgray{k - 2} & & = x(3^{k - 3}) + 3 \\
                    \txtgray{k - 3} & & = x(3^{k - 4}) + 4 \\
                    \quad\txtgray\vdots & & \qquad \vdots
                \end{array}
            \end{rcases}
            x_{k - i + 1} = x(3^{k - i}) + i\ , i \geq 1\ .
        \end{equation*}
        Se usa la relación de recurrencia y el caso base para encontrar el último índice de la serie,
        \begin{equation*}
            \begin{rcases}
                \begin{array}{l l}
                    x(1) = x(3^0) & = 1 \\
                    x(3^0) = x(3^{k - i})
                \end{array}
            \end{rcases}
            \begin{array}{l}
                3^0 = 3^{k - i} \\
                0 = k - i
            \end{array}
            \implies i = k\ .
        \end{equation*}
        El último elemento de la serie será,
        \begin{equation*}
            \begin{array}{l l l}
                \txtgray\vdots & & \qquad\vdots \\
                \txtgray{1} & x(3^k) & = \underbrace{x(3^{k - k})}_{=1} + \underbrace{k}_{\log_{3}(n)}\ .
            \end{array}
        \end{equation*}
        El resultado final coincide con aquel encontrado mediante \textit{forward substitution},
        \begin{equation}
            \result{x(n) = 1 + \log_{3}(n)\ .}
        \end{equation}
    \end{solution}
\end{enumerate}
\end{document}