% -- Document configuration
\documentclass{article}

% -- Input and language settings
% \usepackage[utf8]{inputenc}
\usepackage[spanish]{babel}
\decimalpoint                             % From bable package to use points instead of commas in decimals

% -- Page and line settings
\usepackage{geometry}
\geometry{letterpaper, 
    % margin=2cm, 
    left=2cm, right=2cm,
    top=1.2cm, bottom=1.2cm,
    includefoot, 
    includehead}
\renewcommand{\baselinestretch}{1.2}

% -- Required packages
\usepackage{xcolor}
\usepackage[many]{tcolorbox}
\usepackage{blkarray}                       % Allow matrices with row and column labels
\usepackage{mathtools,amsfonts,amsmath}     % Loads amsmath if not already loaded
\allowdisplaybreaks                         % To allow page breaks if equations are too long
\usepackage[parfill]{parskip}               % No indent and separation lines for paragraphs
\usepackage{cancel}                         % To cancel math terms
\usepackage[shortlabels]{enumitem}          % To handle enumerations
\usepackage{tikz}
\usetikzlibrary{plotmarks}
\usetikzlibrary{math}
\usepackage{pgfplots}
\pgfplotsset{compat=newest}
\usepackage[mode=buildnew]{standalone}      % To import figures in standalone files
\usepackage[hidelinks]{hyperref}
\usepackage[spanish]{cleveref}              % To use autocompleted reference labels, language must be change as in babel package
\usepackage{caption}                        % Caption and subcaption to allow subfigures
\usepackage{subcaption}
\usepackage{float}                          % To specify the location of figures
\usepackage{multicol}                       % To use multicolumns
\usepackage[bottom]{footmisc}               % To locate footnotes at the bottom

% -- Title and heading settings
\usepackage{titling}
\usepackage{fancyhdr}
\pagestyle{fancy}

% -- Code and code formatting
\usepackage{minted}                         % To insert code
\usemintedstyle[julia]{gruvbox-light}       % Code theme and language
\definecolor{bg}{rgb}{0.98, 0.97, 0.88}     % Code block background

\usepackage{textcomp}                       % To allow straight quotes
\usepackage{fontspec}                       % To allow the use of monospace fonts
\setmonofont{JuliaMono}[Path=./codefonts/, Extension=.ttf, UprightFont=*-Regular, ItalicFont=*-RegularItalic, Scale=0.75]

\usepackage{fancyvrb}                       % To change line number font
\renewcommand{\theFancyVerbLine}{\textcolor{gray}{\footnotesize\texttt{\arabic{FancyVerbLine}}}}

\definecolor{light-gray}{gray}{0.95}        % Color, box and style to show small code thingys inside normal text
\newcommand{\code}[1]{\colorbox{light-gray}{\texttt{#1}}}

% -- Bilbiography preferences
\usepackage[square,numbers]{natbib}
\bibliographystyle{unsrt}

% -- Footnotes without numbering
\newcommand\nnfootnote[1]{%
  \begin{NoHyper}
  \renewcommand\thefootnote{}\footnote{#1}%
  \addtocounter{footnote}{-1}%
  \end{NoHyper}
}

% -- Theorems
\newtheorem{theorem}{Theorem}

\lhead{\theinstitution\ -- \thedepartment}
\chead{}
\rhead{\thecourse\ -- \thetitle}
\lfoot{}
\cfoot{\thepage}
\rfoot{}

% -- Problem solution
\newenvironment{solution}
{\begin{quote}
\textbf{Solución:}\medskip

}
{

% \hfill\rule{0.5\textwidth}{0.5pt}
\end{quote}}

% -- Equation result
\newcommand{\result}[1]
{
\tcbhighmath[colframe=white, colback=gray!15, sharp corners]
{#1}
}

% -- Function definitions
\newcommand{\dprod}[2]{{#1} \cdot {#2}}
\newcommand{\txtgray}[1]{\textcolor{gray}{#1}}
\newcommand{\quotes}[1]{\textquotedbl#1\textquotedbl}

% -- Author information
\title{Actividad 6}
\author{Leonardo Flores Torres}
\newcommand\theinstitution{Universidad Veracruzana}
\newcommand\thedepartment{Inteligencia Artificial}
\newcommand\thecourse{Análisis de Algoritmos}

% -- Paths
\newcommand\codepath{../MiniFinder/src/MiniFinder.jl}


% -- Document
\begin{document}

\thispagestyle{empty}

%Title
\begin{center}
\textsc{\theinstitution}\\[2mm]

\thedepartment

\rule{0.6\textwidth}{0.5pt}\\[2mm]

\thecourse \\[4mm]

{\Large \textbf{\thetitle}}\\[2mm]

\theauthor \\[2mm]

{\small \today}
\end{center}
\medskip

% -- 
\vspace{1cm}

\begin{enumerate}
    \item Cálculo de $\pi$ usando el \textit{método de Monte Carlo}.
    \begin{itemize}
        \item Generar los puntos aleatorios usando \textit{convergencia lineal}.
        \item Generar los puntos aleatorios usando \textit{secuencia de Halton}.
        \item Generar los puntos aleatorios usando el generador valores aleatorios de su lenguaje de programación de preferencia.
    \end{itemize}

    \item Para cada caso graficar las curvas de convergencia, hacer al menos $10^6$ iteraciones.
    \item Analizar y determinar qué método fue más preciso.
    \item Pensar cómo aproximar $\pi$ con un número de cifras significativas dadas, por ejemplo, con un número de 4 cifras significativas el valor correspondiente sería $3.14159$.
    \begin{solution}
        La implementación del algoritmo del \textit{generador lineal congruencial} usada en esta actividad está basada en el trabajo hecho por Schlegel \cite{schlegel2008lcg} la cual a su vez está basada en una implementación dentro del estándar del lenguaje de programación \code{C} \cite{saucier2000computer}, siendo esta última referencia de la que se obtienen los parámetros adecuados para este generador.
        

        \begin{figure}[ht!]
            \centering
            \includegraphics[scale=0.5]{../figures/pi_estimation.pdf}
            \caption{xxx}
            % \label{}
        \end{figure}
    \end{solution}
\end{enumerate}

\clearpage
\section*{Apéndice}
\inputminted[
    frame=none,
    autogobble,
    obeytabs=false,
    breaklines,
    tabsize=4,
    linenos=true,
    baselinestretch=1,
    firstnumber=1,
    bgcolor=bg!70,]{julia}{\codepath}

\nocite{*}    % to call all references even if they are not cited in the text
\bibliography{references.bib}

\end{document}

% \begin{minted}[
%     frame=none,
%     autogobble,
%     obeytabs=false,
%     breaklines,
%     tabsize=4,
%     linenos=true,
%     baselinestretch=1,
%     firstnumber=1,
%     bgcolor=bg!70,
%     ]{julia}
%     # Pseudocodigo
%     # Vecino arriba
%     if id > nc
%         agrega (id - nc) como vecino
%     end if
%     # Vecino abajo
%     if id <= nc * (nr - 1)
%         agrega (id + nc) como vecino
%     end if
% \end{minted}