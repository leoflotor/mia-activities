% -- Document configuration
\documentclass{article}

% -- Input and language settings
% \usepackage[utf8]{inputenc}
\usepackage[spanish]{babel}
\decimalpoint                             % From bable package to use points instead of commas in decimals

% -- Page and line settings
\usepackage{geometry}
\geometry{letterpaper, 
    % margin=2cm, 
    left=2cm, right=2cm,
    top=1.2cm, bottom=1.2cm,
    includefoot, 
    includehead}
\renewcommand{\baselinestretch}{1.2}

% -- Required packages
\usepackage{xcolor}
\usepackage[many]{tcolorbox}
\usepackage{blkarray}                       % Allow matrices with row and column labels
\usepackage{mathtools,amsfonts,amsmath}     % Loads amsmath if not already loaded
\allowdisplaybreaks                         % To allow page breaks if equations are too long
\usepackage[parfill]{parskip}               % No indent and separation lines for paragraphs
\usepackage{cancel}                         % To cancel math terms
\usepackage[shortlabels]{enumitem}          % To handle enumerations
\usepackage{tikz}
\usetikzlibrary{plotmarks}
\usetikzlibrary{math}
\usepackage{pgfplots}
\pgfplotsset{compat=newest}
\usepackage[mode=buildnew]{standalone}      % To import figures in standalone files
\usepackage[hidelinks]{hyperref}
\usepackage[spanish]{cleveref}              % To use autocompleted reference labels, language must be change as in babel package
\usepackage{caption}                        % Caption and subcaption to allow subfigures
\usepackage{subcaption}
\usepackage{float}                          % To specify the location of figures
\usepackage{multicol}                       % To use multicolumns
\usepackage[bottom]{footmisc}               % To locate footnotes at the bottom

% -- Title and heading settings
\usepackage{titling}
\usepackage{fancyhdr}
\pagestyle{fancy}

% -- Code and code formatting
\usepackage{minted}                         % To insert code
\usemintedstyle[julia]{gruvbox-light}       % Code theme and language
\definecolor{bg}{rgb}{0.98, 0.97, 0.88}     % Code block background

\usepackage{textcomp}                       % To allow straight quotes
\usepackage{fontspec}                       % To allow the use of monospace fonts
\setmonofont{JuliaMono}[Path=./codefonts/, Extension=.ttf, UprightFont=*-Regular, ItalicFont=*-RegularItalic, Scale=0.75]

\usepackage{fancyvrb}                       % To change line number font
\renewcommand{\theFancyVerbLine}{\textcolor{gray}{\footnotesize\texttt{\arabic{FancyVerbLine}}}}

\definecolor{light-gray}{gray}{0.95}        % Color, box and style to show small code thingys inside normal text
\newcommand{\code}[1]{\colorbox{light-gray}{\texttt{#1}}}

% -- Bilbiography preferences
\usepackage[square,numbers]{natbib}
\bibliographystyle{unsrt}

% -- Footnotes without numbering
\newcommand\nnfootnote[1]{%
  \begin{NoHyper}
  \renewcommand\thefootnote{}\footnote{#1}%
  \addtocounter{footnote}{-1}%
  \end{NoHyper}
}

% -- Theorems
\newtheorem{theorem}{Theorem}

\lhead{\theinstitution\ -- \thedepartment}
\chead{}
\rhead{\thecourse\ -- \thetitle}
\lfoot{}
\cfoot{\thepage}
\rfoot{}

% -- Problem solution
\newenvironment{solution}
{\begin{quote}
\textbf{Solución:}\medskip

}
{

% \hfill\rule{0.5\textwidth}{0.5pt}
\end{quote}}

% -- Equation result
\newcommand{\result}[1]
{
\tcbhighmath[colframe=white, colback=gray!15, sharp corners]
{#1}
}

% -- Function definitions
\newcommand{\dprod}[2]{{#1} \cdot {#2}}
\newcommand{\txtgray}[1]{\textcolor{gray}{#1}}
\newcommand{\quotes}[1]{\textquotedbl#1\textquotedbl}

% -- Author information
\title{Actividad 6}
\author{Leonardo Flores Torres}
\newcommand\theinstitution{Universidad Veracruzana}
\newcommand\thedepartment{Inteligencia Artificial}
\newcommand\thecourse{Análisis de Algoritmos}

% -- Paths
\newcommand\codepath{../MiniFinder/src/MiniFinder.jl}


% -- Document
\begin{document}

\thispagestyle{empty}

%Title
\begin{center}
\textsc{\theinstitution}\\[2mm]

\thedepartment

\rule{0.6\textwidth}{0.5pt}\\[2mm]

\thecourse \\[4mm]

{\Large \textbf{\thetitle}}\\[2mm]

\theauthor \\[2mm]

{\small \today}
\end{center}
\medskip

% -- 
\vspace{1cm}

Para $x(n) = \dprod{3}{x(n-1)} - x(n-2)$, donde $x(0) = 1$ y $x(1) = 3$,
\begin{itemize}
    \item calcular $x(2)$ y $x(3)$,
    \item obtener la fórmula general y la característica,
    \item verificar los resultados de $x(2)$ y $x(3)$ con la fórmula de recurrencia.
\end{itemize}
\begin{solution}
    Primero se usará la relación de recurrencia para obtener los valores de $x(2)$ y $x(3)$:
    \begin{align*}
        x(2) & = \dprod{3}{x(1)} - x(0) & %
        x(3) & = \dprod{3}{x(2)} - x(1) \\
        & = \dprod{3}{3} - 1 & %
        & = \dprod{3}{8} - 3 \\
        & = 8 & % 
        & = 21
    \end{align*}
    Ahora, si se reordenan los términos de $x(n)$ de la siguiente manera
    \begin{equation*}
        x(n) - \dprod{3}{x(n-1)} + x(n-2) = 0, 
    \end{equation*}
    su correspondiente ecuación característica es
    \begin{equation*}
        r^2 - 3 r + 1 = 0\ .
    \end{equation*}
    La cual tiene las siguientes raíces,
    \begin{equation*}
        r_{\pm} = \frac{3 \pm \sqrt{5}}{2}\ .
    \end{equation*}
    Ya que sus raíces son distintas la solución general de $x(n)$ es
    \begin{equation}
        x(n) = \alpha r_{+}^n + \beta r_{-}^n\ . \label{eq:solucion_general}
    \end{equation}
    Posteriormente se aplicaron los casos base $x(1)$ y $x(0)$ para encontrar los valores de las constantes,
    \begin{align}
        x(1) & = \alpha r_{+} + \beta r_{-} = 3\ , \label{eq:caso_base_x1} \\
        x(0) & = \alpha + \beta = 1\ . \label{eq:caso_base_x0}
    \end{align}
    La \cref{eq:caso_base_x0} se resuelve para $\alpha$, y se sustituye en la \cref{eq:caso_base_x1} de donde se obtiene
    \begin{align}
        \beta & = \frac{r_{+} - 3}{r_{+} - r_{-}}\ , \nonumber \\
        & = \frac{5 - 3 \sqrt{5}}{10}\ . \label{eq:beta_valor}
    \end{align}
    Ahora se toma la \cref{eq:beta_valor} y se sustituye en la \cref{eq:caso_base_x1} resolviendo para $\alpha$,
    \begin{align*}
        \alpha & = 1 - \beta\ , \nonumber \\
        & = \frac{5 + 3 \sqrt{5}}{10}\ .
    \end{align*}
    Tomando los valores de $\alpha$ y $\beta$, y sustituyendolos en la \cref{eq:solucion_general} se obtiene la solución al problema,
    \begin{align}
        x(n) & = \left(\frac{5 + 3 \sqrt{5}}{10}\right) \left(\frac{5 + 3 \sqrt{5}}{10}\right)^n + \left(\frac{5 - 3 \sqrt{5}}{10}\right) \left(\frac{5 - 3 \sqrt{5}}{10}\right)^n\ , \nonumber \\
        & = \left(\frac{5 + 3 \sqrt{5}}{10}\right)^{n+1} + \left(\frac{5 - 3 \sqrt{5}}{10}\right)^{n+1}\ . \label{eq:solucion1}
    \end{align}
    Finalmente, se comprueban los resultados de la \cref{eq:solucion1} con los resultados obtenidos al usar la relación de recurrencia,
    \begin{align*}
        x(0) & = \left(\frac{5 + 3 \sqrt{5}}{10}\right) + \left(\frac{5 - 3 \sqrt{5}}{10}\right) = \frac{10}{10} = 1\ , \\
        x(1) & = \left(\frac{5 + 3 \sqrt{5}}{10}\right) \left(\frac{3 + \sqrt{5}}{2}\right) + \left(\frac{5 - 3 \sqrt{5}}{10}\right) \left(\frac{3 + \sqrt{5}}{2}\right) = \frac{6}{2} = 3\ , \\
        x(2) & = \left(\frac{5 + 3 \sqrt{5}}{10}\right) \left(\frac{3 + \sqrt{5}}{2}\right)^2 + \left(\frac{5 - 3 \sqrt{5}}{10}\right) \left(\frac{3 + \sqrt{5}}{2}\right)^2 = \frac{16}{2} = 8\ , \\
        x(3) & = \left(\frac{5 + 3 \sqrt{5}}{10}\right) \left(\frac{3 + \sqrt{5}}{2}\right)^3 + \left(\frac{5 - 3 \sqrt{5}}{10}\right) \left(\frac{3 + \sqrt{5}}{2}\right)^3 = \frac{42}{2} = 21\ .
    \end{align*}
\end{solution}

\nnfootnote{Las soluciones a los problemas se realizaron revisando el libro de referencia \cite{levitin2008introduction}.}

Determinar por inducción matemática que
\begin{equation*}
    \sum_{i=1}^{n} i^2 = \frac{2 n^3 + 3 n^2 + n}{6}\ .
\end{equation*}
\begin{solution}
    Primero, se denominará a la igualdad anterior como $T(n)$, y se dividirá en en dos partes, la del lado derecho y la del lado izquierdo,
    \begin{equation*}
        \begin{split}
            I(n) & = \sum_{i=1}^{k} i^2\ ,\\
            D(n) & = \frac{2 n^3 + 3 n^2 + n}{6}\ .
        \end{split}
    \end{equation*}
    Ambos lados de la igualdad tienen que ser iguales para cualquier valor de $n = k$ donde $k \geq 1$. Se calcularon un par de términos para comprobar este hecho:
    \begin{equation*}
        \begin{array}{l l l}
            \txtgray{k} & I(k) & D(k) \medskip \\
            \txtgray{1} & 1^2 = 1 & \dfrac{2 + 3 + 1}{6} = 1 \medskip \\
            \txtgray{2} & 1^2 + 2^2 = 5 & \dfrac{\dprod{2}{8} + \dprod{3}{4} + 2}{6} = 5 \medskip \\
            \txtgray\vdots & \qquad\vdots & \qquad\vdots
        \end{array}
    \end{equation*}
    Por inspección se puede decir que $I(n) = D(n)$, y se puede proceder con la demostración. Ahora, supóngase que la igualdad $T(n)$ es válida para $n = k$, si eso es cierto entonces queda demostrar que también lo es para $n = k+1$ como se muestra a continuación,
    \begin{equation*}
        \begin{split}
            T(k+1) & = \sum_{i=1}^{k+1} i^2\ ,\\
            & = \sum_{i=1}^{k} i^2 + (k+1)^2\ ,\\
            & = \frac{k+1}{6} \left[2 (k + 1)^2 + 3 (k + 1) + 1\right]\ ,\\
            & = \frac{k+1}{6} \left(2 k^2 + 7 k + 6\right)\ ,\\
            & = \frac{1}{6} \left(2 k^3 + 9 k^2 + 13 k + 6\right)\ ,\\
            T(k + 1) & = \frac{1}{6} \left[\left(2 k^3 + 3 k^2 + k\right) + \left(6 k^2 + 12 k + 6\right)\right]\ ,\\
            \sum_{i=1}^{k} i^2 + (k+1)^2 & = \frac{k}{6} \left(2 k + 1\right) \left(k + 1\right) + \left(k + 1\right)^2\ ,\\
            \sum_{i=1}^{k} i^2 & = \frac{k}{6} \left(2 k + 1\right) \left(k + 1\right)\ ,\\
            & = \frac{2 k^3 + 3^k + 1}{6}\ .
        \end{split}
    \end{equation*}
    Con lo que se demuestra que $T(n)$ también es válida para $n = k+1$.
\end{solution}

\bibliography{references.bib}

\end{document}