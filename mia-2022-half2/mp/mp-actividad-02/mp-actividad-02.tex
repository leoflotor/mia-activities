% -- Document configuration
\documentclass{article}

% -- Input and language settings
% \usepackage[utf8]{inputenc}
\usepackage[spanish]{babel}
\decimalpoint                             % From babel package to use points instead of commas in decimals

% -- Page and line settings
\usepackage{geometry}
\geometry{letterpaper, 
    % margin=2cm, 
    left=3cm, right=3cm,
    top=1.2cm, bottom=1.2cm,
    includefoot, 
    includehead}
\renewcommand{\baselinestretch}{1.2}

% -- Required packages
\usepackage{xcolor}
\usepackage[many]{tcolorbox}
\usepackage{mathtools,amsfonts,amsmath}     % Loads amsmath if not already loaded
\allowdisplaybreaks                         % To allow page breaks if equations are too long
\usepackage[parfill]{parskip}               % No indent and separation lines for paragraphs
\usepackage{cancel}                         % To cancel math terms
\usepackage[shortlabels]{enumitem}          % To handle enumerations
\usepackage{tikz}
\usetikzlibrary{automata, arrows.meta, positioning}
\usepackage[mode=buildnew]{standalone}      % To import figures in standalone files
\usepackage[hidelinks]{hyperref}
\usepackage[spanish]{cleveref}              % To use autocompleted reference labels, language must be change as in babel package
\usepackage{caption}                        % Caption and subcaption to allow subfigures
\usepackage{subcaption}
\usepackage{float}                          % To specify the location of figures
\usepackage{multicol}                       % To use multicolumns
\usepackage[bottom]{footmisc}               % To locate footnotes at the bottom

% -- Title and heading settings
\usepackage{titling}
\usepackage{fancyhdr}
\pagestyle{fancy}

% -- Code and code formatting
\usepackage{minted}                         % To insert code
\usemintedstyle[julia]{gruvbox-light}       % Code theme and language
\definecolor{bg}{rgb}{0.98, 0.97, 0.88}     % Code block background

\usepackage{fontspec}                       % To allow the use of monospace fonts
\setmonofont{JuliaMono}[Path=./codefonts/, Extension=.ttf, UprightFont=*-Regular, ItalicFont=*-RegularItalic, Scale=0.75]

\usepackage{fancyvrb}                       % To change line number font
\renewcommand{\theFancyVerbLine}{\textcolor{gray}{\footnotesize\texttt{\arabic{FancyVerbLine}}}}

\definecolor{light-gray}{gray}{0.95}        % Color, box and style to show small code thingys inside normal text
\newcommand{\code}[1]{\colorbox{light-gray}{\texttt{#1}}}

% -- Bilbiography preferences
\usepackage[square,numbers]{natbib}
\bibliographystyle{unsrt}

% -- Footnotes without numbering
\newcommand\nnfootnote[1]{%
  \begin{NoHyper}
  \renewcommand\thefootnote{}\footnote{#1}%
  \addtocounter{footnote}{-1}%
  \end{NoHyper}
}

% -- Theorems
\newtheorem{theorem}{Theorem}

\lhead{\theinstitution\ -- \thedepartment}
\chead{}
\rhead{Programación para la IA\ -- \thetitle}
\lfoot{}
\cfoot{\thepage}
\rfoot{}

% -- Problem solution
\newenvironment{solution}
{\begin{quote}
\textbf{Solución:}\medskip

}
{

\hfill\rule{0.5\textwidth}{0.5pt}
\end{quote}}

% -- Equation result
\newcommand{\result}[1]
{
\tcbhighmath[colframe=white, colback=gray!15, sharp corners]
{#1}
}

% -- Function definitions
\newcommand{\dprod}[2]{{#1} \cdot {#2}}
\newcommand{\txtgray}[1]{\textcolor{gray}{#1}}

% -- Author information
\title{Actividad 5}
\author{Leonardo Flores Torres}
\newcommand\theinstitution{Universidad Veracruzana}
\newcommand\thedepartment{Inteligencia Artificial}
\newcommand\thecourse{Programación para la Inteligencia Artificial}

% -- Paths
% \newcommand\codelists{../programs/lists.rkt}

% Remove red color boxes of "syntax errors" in minted
\AtBeginEnvironment{minted}{%
  \renewcommand{\fcolorbox}[4][]{#4}}

% -- Document
\begin{document}

\thispagestyle{empty}

%Title
\begin{center}
\textsc{\theinstitution}\\[2mm]

\thedepartment

\rule{0.6\textwidth}{0.5pt}\\[2mm]

\thecourse \\[4mm]

{\Large \textbf{\thetitle}}\\[2mm]

\theauthor \\[2mm]

{\small \today}
\end{center}
\medskip

% -- 
\vspace{1cm}

Suppose that we have three coloured boxes $r$ (red), $b$ (blue), and $g$ (green). Box $r$ contains 3 apples, 4 oranges and 3 limes, box $b$ contains 1 apple, 1 orange and 0 limes, and box $g$ contains 3 apples, 3 oranges and 4 limes. If a box is chosen at random with probabilities $p(r)=0.2$, $p(b)=0.2$, $p(g)=0.6$, and a piece of fruit is removed from the box (with equal probability of selecting any of the items in the box), then what is the probability of selecting an apple? If we observe that the selected fruit is in fact an orange, what is the probability that it came from the green box?

\begin{solution}
    The probabilities of selecting either one of the boxes is
    \begin{equation*}
        \begin{split}
            p(r) & = 2 / 10\ , \\
            p(b) & = 2 / 10\ , \\
            p(g) & = 6 / 10\ .
        \end{split}
    \end{equation*}
    Now, if we assume that we pick one of the boxes at random, then the probability of selecting one of the fruits in that box is just the fraction of that fruit $F$ in the selected box $B$,
    \begin{equation*}
        \begin{aligned}
            p(F=a | B=r) & = 3/10\ ,\\
            p(F=o | B=r) & = 4/10\ ,\\
            p(F=l | B=r) & = 3/10\ ,\\
            p(F=a | B=b) & = 5/10\ ,\\
            p(F=o | B=b) & = 5/10\ ,\\
            p(F=l | B=b) & = 0\ ,\\
            p(F=a | B=g) & = 3/10\ ,\\
            p(F=o | B=g) & = 3/10\ ,\\
            p(F=l | B=g) & = 4/10\ .\\
        \end{aligned}
    \end{equation*}
    The probability of selecting of selecting any of the fruits can be computed with
    \begin{equation*}
        p(X) = \sum\limits_{Y} p(X|Y) p(Y)\ .
    \end{equation*}
    Thus, the probability of selecting an apple is,
    \begin{align*}
        p(F=a) & = p(F=a | B=r)\ p(B=r) + p(F=a | B=b)\ p(B=b) + p(F=a | B=g)\ p(B=g)\ ,\\
        & = (3/10)(2/10) + (5/10)(2/10) + (3/10)(6/10)\ ,\\
        & = \frac{34}{10^2}\ .
    \end{align*}
    Similarly, the probability of selecting an orange is,
    \begin{align*}
        p(F=o) & = p(F=o | B=r)\ p(B=r) + p(F=o | B=b)\ p(B=b) + p(F=o | B=g)\ p(B=g)\ ,\\
        & = (4/10)(2/10) + (5/10)(2/10) + (3/10)(6/10)\ ,\\
        & = \frac{36}{10^2}\ .
    \end{align*}
    And the probability to select a lime is,
    \begin{align*}
        p(F=l) & = p(F=l | B=r)\ p(B=r) + p(F=l | B=b)\ p(B=b) + p(F=l | B=g)\ p(B=g)\ ,\\
        & = (3/10)(2/10) + (0)(2/10) + (4/10)(6/10)\ ,\\
        & = \frac{30}{10^2}\ .
    \end{align*}
    The answer to the first question, \textit{what is the probability of selecting an apple?}, is 
    \begin{equation*}
        \result{
            p(F=a) = 34 / 100\ .
        }
    \end{equation*}
    Now, the probability of selecting a box given that the selected fruit is an orange is,
    \begin{align*}
        p(B=r | F=o) & = \frac{p(F=o | B=r)\ p(B=r)}{P(F=o)} = \frac{(4/10)(2/10)}{(36/10^2)} = \frac{4}{18}\ ,\\
        p(B=b | F=o) & = \frac{p(F=o | B=b)\ p(B=b)}{P(F=o)} = \frac{(5/10)(2/10)}{(36/10^2)} = \frac{5}{18}\ ,\\
        p(B=g | F=o) & = \frac{p(F=o | B=g)\ p(B=g)}{P(F=o)} = \frac{(3/10)(6/10)}{(36/10^2)} = \frac{9}{18}\ .\\
    \end{align*}
    Finally, the answer to the second question, \textit{if we observe that the selected fruit is in fact an orange, what is the probability that it came from the green box?}, is
    \begin{equation*}
        \result{
            p(B=g | F=o) = \frac{9}{18}\ .
        }
    \end{equation*}
\end{solution}

\end{document}