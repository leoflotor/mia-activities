% -- Document configuration
\documentclass{article}

% -- Input and language settings
% \usepackage[utf8]{inputenc}
\usepackage[spanish]{babel}
\decimalpoint                             % From babel package to use points instead of commas in decimals

% -- Page and line settings
\usepackage{geometry}
\geometry{letterpaper, 
    % margin=2cm, 
    left=3cm, right=3cm,
    top=1.2cm, bottom=1.2cm,
    includefoot, 
    includehead}
\renewcommand{\baselinestretch}{1.2}

% -- Required packages
\usepackage{xcolor}
\usepackage[many]{tcolorbox}
\usepackage{mathtools,amsfonts,amsmath}     % Loads amsmath if not already loaded
\allowdisplaybreaks                         % To allow page breaks if equations are too long
\usepackage[parfill]{parskip}               % No indent and separation lines for paragraphs
\usepackage{cancel}                         % To cancel math terms
\usepackage[shortlabels]{enumitem}          % To handle enumerations
\usepackage{tikz}
\usetikzlibrary{automata, arrows.meta, positioning}
\usepackage[mode=buildnew]{standalone}      % To import figures in standalone files
\usepackage[hidelinks]{hyperref}
\usepackage[spanish]{cleveref}              % To use autocompleted reference labels, language must be change as in babel package
\usepackage{caption}                        % Caption and subcaption to allow subfigures
\usepackage{subcaption}
\usepackage{float}                          % To specify the location of figures
\usepackage{multicol}                       % To use multicolumns
\usepackage[bottom]{footmisc}               % To locate footnotes at the bottom

% -- Title and heading settings
\usepackage{titling}
\usepackage{fancyhdr}
\pagestyle{fancy}

% -- Code and code formatting
\usepackage{minted}                         % To insert code
\usemintedstyle[julia]{gruvbox-light}       % Code theme and language
\definecolor{bg}{rgb}{0.98, 0.97, 0.88}     % Code block background

\usepackage{fontspec}                       % To allow the use of monospace fonts
\setmonofont{JuliaMono}[Path=./codefonts/, Extension=.ttf, UprightFont=*-Regular, ItalicFont=*-RegularItalic, Scale=0.75]

\usepackage{fancyvrb}                       % To change line number font
\renewcommand{\theFancyVerbLine}{\textcolor{gray}{\footnotesize\texttt{\arabic{FancyVerbLine}}}}

\definecolor{light-gray}{gray}{0.95}        % Color, box and style to show small code thingys inside normal text
\newcommand{\code}[1]{\colorbox{light-gray}{\texttt{#1}}}

% -- Bilbiography preferences
\usepackage[square,numbers]{natbib}
\bibliographystyle{unsrt}

% -- Footnotes without numbering
\newcommand\nnfootnote[1]{%
  \begin{NoHyper}
  \renewcommand\thefootnote{}\footnote{#1}%
  \addtocounter{footnote}{-1}%
  \end{NoHyper}
}

% -- Theorems
\newtheorem{theorem}{Theorem}

\lhead{\theinstitution\ -- \thedepartment}
\chead{}
\rhead{Programación para la IA\ -- \thetitle}
\lfoot{}
\cfoot{\thepage}
\rfoot{}

% -- Problem solution
\newenvironment{solution}
{\begin{quote}
\textbf{Solución:}\medskip

}
{

\hfill\rule{0.5\textwidth}{0.5pt}
\end{quote}}

% -- Equation result
\newcommand{\result}[1]
{
\tcbhighmath[colframe=white, colback=gray!15, sharp corners]
{#1}
}

% -- Function definitions
\newcommand{\dprod}[2]{{#1} \cdot {#2}}
\newcommand{\txtgray}[1]{\textcolor{gray}{#1}}

% -- Author information
\title{Actividad 5}
\author{Leonardo Flores Torres}
\newcommand\theinstitution{Universidad Veracruzana}
\newcommand\thedepartment{Inteligencia Artificial}
\newcommand\thecourse{Programación para la Inteligencia Artificial}

% -- Paths
% \newcommand\codelists{../programs/lists.rkt}

% Remove red color boxes of "syntax errors" in minted
\AtBeginEnvironment{minted}{%
  \renewcommand{\fcolorbox}[4][]{#4}}

% -- Document
\begin{document}

\thispagestyle{empty}

%Title
\begin{center}
\textsc{\theinstitution}\\[2mm]

\thedepartment

\rule{0.6\textwidth}{0.5pt}\\[2mm]

\thecourse \\[4mm]

{\Large \textbf{\thetitle}}\\[2mm]

\theauthor \\[2mm]

{\small \today}
\end{center}
\medskip

% -- 
\vspace{1cm}

Elabore un programa en \code{prolog} con el espacio de estados que comienza con el número 1 y la función sucesor para el estado $n$ devuelve 2 estados, los números $2n$ y $2n + 1$.
\begin{itemize}
    \item Dibuje la porción de estados para los estados del 1 al 15. Supongamos que el estado meta es el 11.
    \item Programe los predicados con las reglas necesarias para encontrar el orden en que serán visitados los nodos en profundidad y amplitud.
\end{itemize}
\begin{solution}
    \begin{figure}[ht!]
        \centering
        \begin{tikzpicture}[
            nodegrey/.style={circle, draw=gray!90, fill=gray!10, very thick, minimum size=7mm},
            nodegreen/.style={circle, draw=green!90, fill=green!10, very thick, minimum size=7mm}]
            \node[nodegreen] (a1) at (0,0) {$1$};
            \node[nodegrey] (a2) at (-1,-1) {$2$};
            \node[nodegrey] (a3) at (1,-1) {$3$};
            \node[nodegrey] (a4) at (-2,-2.25) {$4$};
            \node[nodegrey] (a5) at (-0.8,-2.25) {$5$};
            \node[nodegrey] (a6) at (0.8,-2.25) {$6$};
            \node[nodegrey] (a7) at (2,-2.25) {$7$};
            \node[nodegrey] (a8) at (-2.5,-3.5) {$8$};
            \node[nodegrey] (a9) at (-3.5,-3.5) {$9$};
            \node[nodegrey] (a10) at (-0.5,-3.5) {$10$};
            \node[nodegreen] (a11) at (-1.5,-3.5) {$11$};
            \node[nodegrey] (a12) at (0.5,-3.5) {$12$};
            \node[nodegrey] (a13) at (1.5,-3.5) {$13$};
            \node[nodegrey] (a14) at (2.5,-3.5) {$14$};
            \node[nodegrey] (a15) at (3.5,-3.5) {$15$};
            \draw[arrows=->] (a1) -- (a2);
            \draw[arrows=->] (a1) -- (a3);
            \draw[arrows=->] (a2) -- (a4);
            \draw[arrows=->] (a2) -- (a5);
            \draw[arrows=->] (a3) -- (a6);
            \draw[arrows=->] (a3) -- (a7);
            \draw[arrows=->] (a4) -- (a8);
            \draw[arrows=->] (a4) -- (a9);
            \draw[arrows=->] (a5) -- (a10);
            \draw[arrows=->] (a5) -- (a11);
            \draw[arrows=->] (a6) -- (a12);
            \draw[arrows=->] (a6) -- (a13);
            \draw[arrows=->] (a7) -- (a14);
            \draw[arrows=->] (a7) -- (a15);
        \end{tikzpicture}
    \end{figure}

    Para resolver el caso de búsqueda en profundidad primero se declararon las conexiones entre nodos como se muestra a continuación:
        \begin{minted}[
            frame=none,
            autogobble,
            obeytabs=false,
            breaklines,
            tabsize=4,
            linenos=true,
            baselinestretch=1,
            firstnumber=last,
            bgcolor=bg!70,
            ]{prolog}
            % Conexiones entre nodos
            conexion(inicio,1).
            conexion(1,2).
            conexion(1,3).
            conexion(2,4).
            conexion(2,5).
            conexion(3,6).
            conexion(3,7).
            conexion(4,8).
            conexion(4,9).
            conexion(5,10).
            conexion(5,11).
            conexion(6,12).
            conexion(6,13).
            conexion(7,14).
            conexion(7,15).
            conexion(11,fin).
        \end{minted}
        No es necesario declarar la regla \code{conexion(inicio,1)} ni \code{conexion(11,fin)}, se podría cambiar las ocurrencias de \code{1} por \code{inicio} y las de \code{11} por \code{fin},
        \begin{itemize}
            \item \code{conexion(inicio,2)},
            \item \code{conexion(inicio,3)},
            \item \code{conexion(5,fin)}.
        \end{itemize}
        Lo que se logra al haber escrito las conexiones como se hizo es el definir una conexión auxiliar que apunta a los nodos de inicio y de término, \code{1} y \code{11}, respectivamente. Posteriormente se define la meta para indicar cuando se ha alcanzado nuestro nodo objetivo:
        \begin{minted}[
            frame=none,
            autogobble,
            obeytabs=false,
            breaklines,
            tabsize=4,
            linenos=true,
            baselinestretch=1,
            firstnumber=last,
            bgcolor=bg!70,
            ]{prolog}
            %%% Declaracion de la meta
            meta(fin).
        \end{minted}
        
        Con la meta y las conexiones declaradas es necesario definir las reglas que indiquen las relaciones entre nodos, esto es, permitir a \code{prolog} identificar como un nodo está conectado con otro:
        \begin{minted}[
            frame=none,
            autogobble,
            obeytabs=false,
            breaklines,
            tabsize=4,
            linenos=true,
            baselinestretch=1,
            firstnumber=last,
            bgcolor=bg!70,
            ]{prolog}
            %%% Reglas para definir las condiciones de como dos nodos estan conectados
            sucesor(Posicion1, Posicion2) :- conexion(Posicion1, Posicion2).
            sucesor(Posicion1, Posicion2) :- conexion(Posicion2, Posicion1).
        \end{minted}

        Finalmente se agrega el algoritmo de búsqueda en profundidad:
        \begin{minted}[
            frame=none,
            autogobble,
            obeytabs=false,
            breaklines,
            tabsize=4,
            linenos=true,
            baselinestretch=1,
            firstnumber=last,
            bgcolor=bg!70,
            ]{prolog}
            %%% Algoritmo de busqueda en profundidad
            ruta([fin|RestoDeRuta], [fin|RestoDeRuta]).
            ruta([PosicionActual|RestoDeRuta], Sol) :-
                sucesor(PosicionActual, PosicionSiguiente),
                not(member(PosicionSiguiente, RestoDeRuta)),
                ruta([PosicionSiguiente,PosicionActual|RestoDeRuta], Sol).
        \end{minted}

        Si se permite a \code{prolog} mostrar los pasos que realiza para ejecutar el algoritmo se podrá observar que efectivamente, la búsqueda se efectua en profundidad, esto se activa con el comando \code{trace.}, y \code{notrace.} si no se quiere mostrar ya los pasos de ejecución.

        Para requerir la solución de la búsqueda se escribió la siguiente regla:
        \begin{minted}[
            frame=none,
            autogobble,
            obeytabs=false,
            breaklines,
            tabsize=4,
            linenos=true,
            baselinestretch=1,
            firstnumber=last,
            bgcolor=bg!70,
            ]{prolog}
            %%% Solicitar solucion
            solucion(Inicio, Sol) :-
                ruta([Inicio], SolAux),
                reverse(SolAux, Sol).
        \end{minted}
        
        Al ejecutar \code{solucion} con el punto de partida al punto de llegada deseados en \code{prolog}, en la terminal, se obtiene como resultado lo siguiente:
        \begin{minted}[
            frame=none,
            autogobble,
            obeytabs=false,
            breaklines,
            tabsize=4,
            linenos=true,
            baselinestretch=1,
            firstnumber=1,
            bgcolor=bg!70,
            numberblanklines=false,
            ]{prolog}
            ?- [searchDepth].
            true.

            ?- solucion(inicio,S).
            S = [inicio, 1, 2, 5, 11, fin] ;
            false.
        \end{minted}

        Ahora, para el caso de búsqueda en amplitud primero se declaran las conexiones que existen como en el caso anterior:
        \begin{minted}[
            frame=none,
            autogobble,
            obeytabs=false,
            breaklines,
            tabsize=4,
            linenos=true,
            baselinestretch=1,
            firstnumber=1,
            bgcolor=bg!70,
            ]{prolog}
            % Conexiones entre nodos
            conexion(1,2).
            conexion(1,3).
            conexion(2,4).
            conexion(2,5).
            conexion(3,6).
            conexion(3,7).
            conexion(4,8).
            conexion(4,9).
            conexion(5,10).
            conexion(5,11).
            conexion(6,12).
            conexion(6,13).
            conexion(7,14).
            conexion(7,15).
        \end{minted}

        Las conexiones en realidad son las mismas, la única diferencia que se hizo aquí fue remover \code{inicio} y \code{fin} para demostrar que se puede hacer la búsqueda sin esas conexiones auxiliares. La meta sigue siendo la misma, solamente se sustituye \code{fin} por \code{11}:
        \begin{minted}[
            frame=none,
            autogobble,
            obeytabs=false,
            breaklines,
            tabsize=4,
            linenos=true,
            baselinestretch=1,
            firstnumber=last,
            bgcolor=bg!70,
            ]{prolog}
            %%% Declaracion de la meta
            meta(11).
        \end{minted}

        Ahora, el algoritmo para la búsqueda en amplitud es el siguiente:
        \begin{minted}[
            frame=none,
            autogobble,
            obeytabs=false,
            breaklines,
            tabsize=4,
            linenos=true,
            baselinestretch=1,
            firstnumber=last,
            bgcolor=bg!70,
            ]{prolog}
            %%% La ruta en realidad es una lista de caminos que se van creando
            %%% conforme se visitan los nodos en amplitud
            ruta([[NodoActual|Camino]|_], [NodoActual|Camino]) :-
                meta(NodoActual).
            ruta([Camino|Caminos], Sol) :-
                extender(Camino, NuevosCaminos),
                append(Caminos, NuevosCaminos, Caminos1),
                ruta(Caminos1, Sol).

            extender([NodoActual|Camino], NuevosCaminos) :-
                bagof([NuevoNodo,NodoActual|Camino],
                    (conexion(NodoActual, NuevoNodo),
                    not((member(NuevoNodo, [NodoActual|Camino])))),
                    NuevosCaminos),
                    !.
            %%% Caso si ya no hay sucesores en Camino
            extender(_, []).
        \end{minted}

        De manera similar al caso anterior se define una regla \code{solucion} para encontrar la solución de la búsqueda:
        \begin{minted}[
            frame=none,
            autogobble,
            obeytabs=false,
            breaklines,
            tabsize=4,
            linenos=true,
            baselinestretch=1,
            firstnumber=last,
            bgcolor=bg!70,
            ]{prolog}
            %%% Búsqueda primero en Amplitud
            solucion(Inicio, Sol) :-
                ruta([[Inicio]], SolAux),
                reverse(SolAux, Sol).
        \end{minted}
        
        Y realizando la búsqueda en terminal se obtiene el siguiente resultado:
        \begin{minted}[
            frame=none,
            autogobble,
            obeytabs=false,
            breaklines,
            tabsize=4,
            linenos=true,
            baselinestretch=1,
            firstnumber=1,
            bgcolor=bg!70,
            ]{prolog}
            ?- [searchAmp].
            true.

            ?- solucion(1,S).
            S = [1, 2, 5, 11] ;
            false.
        \end{minted}
\end{solution}

\nocite{*}
\bibliography{references.bib}

\end{document}
