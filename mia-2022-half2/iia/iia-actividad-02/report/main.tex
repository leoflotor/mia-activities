% -- Document configuration
\documentclass{article}

% -- Input and language settings
% \usepackage[utf8]{inputenc}
\usepackage[spanish]{babel}
\decimalpoint                             % From bable package to use points instead of commas in decimals

% -- Page and line settings
\usepackage{geometry}
\geometry{letterpaper, 
    % margin=2cm, 
    left=2cm, right=2cm,
    top=1.2cm, bottom=1.2cm,
    includefoot, 
    includehead}
\renewcommand{\baselinestretch}{1.2}

% -- Required packages
\usepackage{xcolor}
\usepackage[many]{tcolorbox}
\usepackage{blkarray}                       % Allow matrices with row and column labels
\usepackage{mathtools,amsfonts,amsmath}     % Loads amsmath if not already loaded
\allowdisplaybreaks                         % To allow page breaks if equations are too long
\usepackage[parfill]{parskip}               % No indent and separation lines for paragraphs
\usepackage{cancel}                         % To cancel math terms
\usepackage[shortlabels]{enumitem}          % To handle enumerations
\usepackage{tikz}
\usetikzlibrary{plotmarks}
\usetikzlibrary{math}
\usepackage{pgfplots}
\pgfplotsset{compat=newest}
\usepackage[mode=buildnew]{standalone}      % To import figures in standalone files
\usepackage[hidelinks]{hyperref}
\usepackage[spanish]{cleveref}              % To use autocompleted reference labels, language must be change as in babel package
\usepackage{caption}                        % Caption and subcaption to allow subfigures
\usepackage{subcaption}
\usepackage{float}                          % To specify the location of figures
\usepackage{multicol}                       % To use multicolumns
\usepackage[bottom]{footmisc}               % To locate footnotes at the bottom

% -- Title and heading settings
\usepackage{titling}
\usepackage{fancyhdr}
\pagestyle{fancy}

% -- Code and code formatting
\usepackage{minted}                         % To insert code
\usemintedstyle[julia]{gruvbox-light}       % Code theme and language
\definecolor{bg}{rgb}{0.98, 0.97, 0.88}     % Code block background

\usepackage{textcomp}                       % To allow straight quotes
\usepackage{fontspec}                       % To allow the use of monospace fonts
\setmonofont{JuliaMono}[Path=./codefonts/, Extension=.ttf, UprightFont=*-Regular, ItalicFont=*-RegularItalic, Scale=0.75]

\usepackage{fancyvrb}                       % To change line number font
\renewcommand{\theFancyVerbLine}{\textcolor{gray}{\footnotesize\texttt{\arabic{FancyVerbLine}}}}

\definecolor{light-gray}{gray}{0.95}        % Color, box and style to show small code thingys inside normal text
\newcommand{\code}[1]{\colorbox{light-gray}{\texttt{#1}}}

% -- Bilbiography preferences
\usepackage[square,numbers]{natbib}
\bibliographystyle{unsrt}

% -- Footnotes without numbering
\newcommand\nnfootnote[1]{%
  \begin{NoHyper}
  \renewcommand\thefootnote{}\footnote{#1}%
  \addtocounter{footnote}{-1}%
  \end{NoHyper}
}

% -- Theorems
\newtheorem{theorem}{Theorem}

\lhead{\theinstitution\ -- \thedepartment}
\chead{}
\rhead{\thecourse\ -- \thetitle}
\lfoot{}
\cfoot{\thepage}
\rfoot{}

% -- Problem solution
\newenvironment{solution}
{\begin{quote}
\textbf{Solución:}\medskip

}
{

% \hfill\rule{0.5\textwidth}{0.5pt}
\end{quote}}

% -- Equation result
\newcommand{\result}[1]
{
\tcbhighmath[colframe=white, colback=gray!15, sharp corners]
{#1}
}

% -- Function definitions
\newcommand{\dprod}[2]{{#1} \cdot {#2}}
\newcommand{\txtgray}[1]{\textcolor{gray}{#1}}
\newcommand{\quotes}[1]{\textquotedbl#1\textquotedbl}

% -- Author information
\title{Actividad 6}
\author{Leonardo Flores Torres}
\newcommand\theinstitution{Universidad Veracruzana}
\newcommand\thedepartment{Inteligencia Artificial}
\newcommand\thecourse{Análisis de Algoritmos}

% -- Paths
\newcommand\codepath{../MiniFinder/src/MiniFinder.jl}


% -- Document
\begin{document}

\thispagestyle{empty}

%Title
\begin{center}
\textsc{\theinstitution}\\[2mm]

\thedepartment

\rule{0.6\textwidth}{0.5pt}\\[2mm]

\thecourse \\[4mm]

{\Large \textbf{\thetitle}}\\[2mm]

\theauthor \\[2mm]

{\small \today}
\end{center}
\medskip

% -- 
\vspace{1cm}

Elabore un programa en \code{prolog} con el espacio de estados que comienza con el número 1 y la función sucesor para el estado $n$ devuelve 2 estados, los números $2n$ y $2n + 1$.
\begin{itemize}
    \item Dibuje la porción de estados para los estados del 1 al 15. Supongamos que el estado meta es el 11.
    \item Programe los predicados con las reglas necesarias para encontrar el orden en que serán visitados los nodos en profundidad y amplitud.
\end{itemize}
\begin{solution}
    \begin{figure}[ht!]
        \centering
        \begin{tikzpicture}[
            nodegrey/.style={circle, draw=gray!90, fill=gray!10, very thick, minimum size=7mm},
            nodegreen/.style={circle, draw=green!90, fill=green!10, very thick, minimum size=7mm}]
            \node[nodegreen] (a1) at (0,0) {$1$};
            \node[nodegrey] (a2) at (-1,-1) {$2$};
            \node[nodegrey] (a3) at (1,-1) {$3$};
            \node[nodegrey] (a4) at (-2,-2.25) {$4$};
            \node[nodegrey] (a5) at (-0.8,-2.25) {$5$};
            \node[nodegrey] (a6) at (0.8,-2.25) {$6$};
            \node[nodegrey] (a7) at (2,-2.25) {$7$};
            \node[nodegrey] (a8) at (-2.5,-3.5) {$8$};
            \node[nodegrey] (a9) at (-3.5,-3.5) {$9$};
            \node[nodegrey] (a10) at (-0.5,-3.5) {$10$};
            \node[nodegreen] (a11) at (-1.5,-3.5) {$11$};
            \node[nodegrey] (a12) at (0.5,-3.5) {$12$};
            \node[nodegrey] (a13) at (1.5,-3.5) {$13$};
            \node[nodegrey] (a14) at (2.5,-3.5) {$14$};
            \node[nodegrey] (a15) at (3.5,-3.5) {$15$};
            \draw[arrows=->] (a1) -- (a2);
            \draw[arrows=->] (a1) -- (a3);
            \draw[arrows=->] (a2) -- (a4);
            \draw[arrows=->] (a2) -- (a5);
            \draw[arrows=->] (a3) -- (a6);
            \draw[arrows=->] (a3) -- (a7);
            \draw[arrows=->] (a4) -- (a8);
            \draw[arrows=->] (a4) -- (a9);
            \draw[arrows=->] (a5) -- (a10);
            \draw[arrows=->] (a5) -- (a11);
            \draw[arrows=->] (a6) -- (a12);
            \draw[arrows=->] (a6) -- (a13);
            \draw[arrows=->] (a7) -- (a14);
            \draw[arrows=->] (a7) -- (a15);
        \end{tikzpicture}
    \end{figure}

    Para resolver el caso de búsqueda en profundidad primero se declararon las conexiones entre nodos
        \begin{minted}[
            frame=none,
            autogobble,
            obeytabs=false,
            breaklines,
            tabsize=4,
            linenos=true,
            baselinestretch=1,
            firstnumber=last,
            bgcolor=bg!70,
            ]{prolog}
            % Conexiones entre nodos
            conexion(inicio,1).
            conexion(1,2).
            conexion(1,3).
            conexion(2,4).
            conexion(2,5).
            conexion(3,6).
            conexion(3,7).
            conexion(4,8).
            conexion(4,9).
            conexion(5,10).
            conexion(5,11).
            conexion(6,12).
            conexion(6,13).
            conexion(7,14).
            conexion(7,15).
            conexion(11,fin).
        \end{minted}
        \begin{minted}[
            frame=none,
            autogobble,
            obeytabs=false,
            breaklines,
            tabsize=4,
            linenos=true,
            baselinestretch=1,
            firstnumber=last,
            bgcolor=bg!70,
            ]{prolog}
            %%% Declaracion de la meta
            meta(fin).
        \end{minted}
\end{solution}

% \begin{minted}[
%     frame=none,
%     autogobble,
%     obeytabs=false,
%     breaklines,
%     tabsize=4,
%     linenos=true,
%     % numbersep=-10pt,
%     baselinestretch=1,
%     firstnumber=1,
%     bgcolor=bg!70,
%     ]{prolog}
% \end{minted}

% \inputminted[
%     frame=none,
%     autogobble,
%     obeytabs=false,
%     breaklines,
%     tabsize=4,
%     linenos=true,
%     % numbersep=-10pt,
%     baselinestretch=1,
%     firstnumber=1,
%     bgcolor=bg!70,]{prolog}{\codesolfirst}

% \vspace{1cm}
% \pagebreak

\nocite{*}
\bibliography{references.bib}

\end{document}
