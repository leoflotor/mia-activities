% -- Document configuration
\documentclass{article}

% -- Input and language settings
% \usepackage[utf8]{inputenc}
\usepackage[spanish]{babel}
\decimalpoint                             % From bable package to use points instead of commas in decimals

% -- Page and line settings
\usepackage{geometry}
\geometry{letterpaper, 
    % margin=2cm, 
    left=2cm, right=2cm,
    top=1.2cm, bottom=1.2cm,
    includefoot, 
    includehead}
\renewcommand{\baselinestretch}{1.2}

% -- Required packages
\usepackage{xcolor}
\usepackage[many]{tcolorbox}
\usepackage{blkarray}                       % Allow matrices with row and column labels
\usepackage{mathtools,amsfonts,amsmath}     % Loads amsmath if not already loaded
\allowdisplaybreaks                         % To allow page breaks if equations are too long
\usepackage[parfill]{parskip}               % No indent and separation lines for paragraphs
\usepackage{cancel}                         % To cancel math terms
\usepackage[shortlabels]{enumitem}          % To handle enumerations
\usepackage{tikz}
\usetikzlibrary{plotmarks}
\usetikzlibrary{math}
\usepackage{pgfplots}
\pgfplotsset{compat=newest}
\usepackage[mode=buildnew]{standalone}      % To import figures in standalone files
\usepackage[hidelinks]{hyperref}
\usepackage[spanish]{cleveref}              % To use autocompleted reference labels, language must be change as in babel package
\usepackage{caption}                        % Caption and subcaption to allow subfigures
\usepackage{subcaption}
\usepackage{float}                          % To specify the location of figures
\usepackage{multicol}                       % To use multicolumns
\usepackage[bottom]{footmisc}               % To locate footnotes at the bottom

% -- Title and heading settings
\usepackage{titling}
\usepackage{fancyhdr}
\pagestyle{fancy}

% -- Code and code formatting
\usepackage{minted}                         % To insert code
\usemintedstyle[julia]{gruvbox-light}       % Code theme and language
\definecolor{bg}{rgb}{0.98, 0.97, 0.88}     % Code block background

\usepackage{textcomp}                       % To allow straight quotes
\usepackage{fontspec}                       % To allow the use of monospace fonts
\setmonofont{JuliaMono}[Path=./codefonts/, Extension=.ttf, UprightFont=*-Regular, ItalicFont=*-RegularItalic, Scale=0.75]

\usepackage{fancyvrb}                       % To change line number font
\renewcommand{\theFancyVerbLine}{\textcolor{gray}{\footnotesize\texttt{\arabic{FancyVerbLine}}}}

\definecolor{light-gray}{gray}{0.95}        % Color, box and style to show small code thingys inside normal text
\newcommand{\code}[1]{\colorbox{light-gray}{\texttt{#1}}}

% -- Bilbiography preferences
\usepackage[square,numbers]{natbib}
\bibliographystyle{unsrt}

% -- Footnotes without numbering
\newcommand\nnfootnote[1]{%
  \begin{NoHyper}
  \renewcommand\thefootnote{}\footnote{#1}%
  \addtocounter{footnote}{-1}%
  \end{NoHyper}
}

% -- Theorems
\newtheorem{theorem}{Theorem}

\lhead{\theinstitution\ -- \thedepartment}
\chead{}
\rhead{\thecourse\ -- \thetitle}
\lfoot{}
\cfoot{\thepage}
\rfoot{}

% -- Problem solution
\newenvironment{solution}
{\begin{quote}
\textbf{Solución:}\medskip

}
{

% \hfill\rule{0.5\textwidth}{0.5pt}
\end{quote}}

% -- Equation result
\newcommand{\result}[1]
{
\tcbhighmath[colframe=white, colback=gray!15, sharp corners]
{#1}
}

% -- Function definitions
\newcommand{\dprod}[2]{{#1} \cdot {#2}}
\newcommand{\txtgray}[1]{\textcolor{gray}{#1}}
\newcommand{\quotes}[1]{\textquotedbl#1\textquotedbl}

% -- Author information
\title{Actividad 6}
\author{Leonardo Flores Torres}
\newcommand\theinstitution{Universidad Veracruzana}
\newcommand\thedepartment{Inteligencia Artificial}
\newcommand\thecourse{Análisis de Algoritmos}

% -- Paths
\newcommand\codepath{../MiniFinder/src/MiniFinder.jl}


% -- Document
\begin{document}

\thispagestyle{empty}

% Title
\begin{center}
\textsc{\theinstitution}\\[2mm]

\thedepartment

\rule{0.6\textwidth}{0.5pt}\\[2mm]

\thecourse \\[4mm]

{\Large \textbf{\thetitle}}\\[2mm]

\theauthor \\[2mm]

{\small \today}
\end{center}
\medskip

% --
\vspace{1cm}

Convert the NFA shown in \cref{fig:nondet-automaton}, corresponding to the machine $N$, to a DFA:
\begin{figure}[ht!]
    \centering
    \includestandalone[mode=buildnew]{nondet-automaton}
    \caption{The NFA $N$.}
    \label{fig:nondet-automaton}
\end{figure}
\begin{solution}
    The automaton $N$ is described by the 5-tuple $\left(Q, \Sigma, \delta, q_1, F\right)$. Its set of states $Q = \{q_1, q_2, q_3, q_4\}$, its alphabet $\Sigma = \{0, 1\}$, the start state is $q_1$, and the set of accept states is $F = \{q_4\}$. Additionally, $N$'s transition table is shown below:
    \begin{equation*}
        \begin{array}{c | c c}
            Q \backslash \Sigma & 0 & 1 \\
            \hline
            q_1 & \{q_1\} & \{q_1, q_2\} \\
            q_2 & \{q_3\} & \emptyset \\
            q_3 & \emptyset & \{q_4\} \\
            q_4 & \{q_4\} & \{q_4\}
        \end{array}
    \end{equation*}

    Now, to find a deterministic equivalent $M$ of $N$ it is necessary to compute the 5-tuple given by $(Q', \Sigma, \delta', q'_{1}, F')$. The set of states of the deterministic automaton $Q'$ correspond to all the possible states product of the power set of $Q$, given by $\mathbb{P}(Q)$,
    \begin{equation*}
        \begin{split}
            Q' & = \{\emptyset, \{q_1\}, \{q_2\}, \{q_3\}, \{q_4\}, \\
            & \qquad \{q_1, q_2\}, \{q_1, q_3\}, \{q_1, q_4\}, \{q_2, q_3\}, \{q_2, q_4\}, \{q_3, q_4\}, \\
            & \qquad \{q_1, q_2, q_3\}, \{q_1, q_2, q_4\}, \{q_1, q_3, q_4\}, \{q_2, q_3, q_4\}, \\ 
            & \qquad \{q_1, q_2, q_3, q_4\}\}\ ,\\
            & = \{\emptyset, R_{1}, R_{2}, R_{3}, R_{4}, \\
            & \qquad R_{12}, R_{13}, R_{14}, R_{23}, R_{24}, R_{34}, \\
            & \qquad R_{123}, R_{124}, R_{134}, R_{234}, \\
            & \qquad R_{1234}\}\ .
        \end{split}
    \end{equation*}
    The notation $R_{i,j,k,l}$ is used here for simplicity.

    It can be observed that $Q'$ is a set of all possible subsets of $Q$. But, as it will be shown later, not all subsets are \textit{reachables} The transition function for the DFA equivalence is $\delta'(R,a) = \{q \in Q | q \in \delta(r,a) \text{ for some } r \in R\}$ which can also be interpreted as 
    \begin{equation*}
        \delta'(R,a) = \bigcup\limits_{r \in R} \delta(r,a)\ .
    \end{equation*}
    The transitions $\delta'(R,a)$ of $M$ are shown below:
    \begin{align*}
        \delta'(R_{1}, 0) & = R_{1} & % 
            \delta'(R_{24}, 0) & = R_{3} \cup R_{4} = R_{34} \\
        \delta'(R_{1}, 1) & = R_{12} & % 
            \delta'(R_{24}, 1) & = \emptyset \cup R_{4} = R_{4} \\
        \delta'(R_{2}, 0) & = R_{3} & % 
            \delta'(R_{34}, 0) & = \emptyset \cup R_{4} = R_{4} \\
        \delta'(R_{2}, 1) & = \emptyset & % 
            \delta'(R_{34}, 1) & = R_{4} \cup R_{4} = R_{4} \\
        \delta'(R_{3}, 0) & = \emptyset & % 
            \delta'(R_{123}, 0) & = R_{1} \cup R_{3} \cup \emptyset = R_{13} \\
        \delta'(R_{3}, 1) & = R_{4} & % 
            \delta'(R_{123}, 1) & = R_{12} \cup \emptyset \cup R_{4} = R_{124} \\
        \delta'(R_{4}, 0) & = R_{4} & % 
            \delta'(R_{124}, 0) & = R_{1} \cup R_{3} \cup R_{4} = R_{134} \\
        \delta'(R_{4}, 1) & = R_{4} & % 
            \delta'(R_{124}, 1) & = R_{12} \cup \emptyset \cup R_{4} = R_{124} \\
        \delta'(R_{12}, 0) & = R_{13} & % 
            \delta'(R_{134}, 0) & = R_{1} \cup \emptyset \cup R_{4} = R_{14} \\
        \delta'(R_{12}, 1) & = R_{12} \cup R_{12} = R_{12} & % 
            \delta'(R_{134}, 1) & = R_{12} \cup R_{4} \cup R_{4} = R_{124} \\
        \delta'(R_{13}, 0) & = R_{1} \cup \emptyset = R_{1} & % 
            \delta'(R_{234}, 0) & = R_{3} \cup \emptyset \cup R_{4} = R_{34} \\
        \delta'(R_{13}, 1) & = R_{12} \cup R_{4} = R_{124} & % 
            \delta'(R_{234}, 1) & = \emptyset \cup R_{4} \cup R_{4} = R_{4} \\
        \delta'(R_{14}, 0) & = R_{1} \cup R_{4} = R_{14} & % 
            \delta'(R_{1234}, 0) & = R_{1} \cup R_{3} \cup \emptyset \cup R_{4}= R_{134} \\
        \delta'(R_{14}, 1) & = R_{12} \cup R_{4} = R_{124} & % 
            \delta'(R_{1234}, 1) & = R_{12} \cup \emptyset \cup R_{4} \cup R_{4} = R_{124} \\
        \delta'(R_{23}, 0) & = R_{3} \cup \emptyset = R_{3} & %
            \delta'(\emptyset, 0) = \emptyset \\
        \delta'(R_{23}, 1) & = \emptyset \cup R_{4} = R_{4} & %
            \delta'(\emptyset, 1) = \emptyset
    \end{align*}
    The above transitions were included only to shown how the union operation from the definition of $\delta'$ is supposed to operate but it was done directly with the $R$ subsets instead of their elements $r \in R$. Additionally, the transitions above can be written in a transition table for better readability:
    \begin{equation*}
        \begin{array}{c | l l}
            Q' \backslash \Sigma & 0 & 1 \\
            \hline
            \emptyset & \emptyset & \emptyset \\
            R_{1} & R_{1} & R_{12} \\
            R_{2} & R_{3} & \emptyset \\
            R_{3} & \emptyset & R_{4} \\
            R_{4} & R_{4} & R_{4} \\
            R_{12} & R_{13} & R_{12} \\
            R_{13} & R_{1} & R_{124} \\
            R_{14} & R_{14} & R_{124} \\
            R_{23} & R_{3} & R_{4} \\
            R_{24} & R_{34} & R_{4} \\
            R_{34} & R_{4} & R_{4} \\
            R_{123} & R_{13} & R_{124} \\
            R_{124} & R_{134} & R_{124} \\
            R_{134} & R_{14} & R_{124} \\
            R_{234} & R_{34} & R_{4} \\
            R_{1234} & R_{134} & R_{124}
        \end{array}
    \end{equation*}

    It is important to notice that in the definition of the equivalence of a NFA to a DFA, the start state needs to be the same for both cases. Thus, the execution path for $M$ shall commence in $R_1$ and it will follow from there. This means that any other state in $Q'$ that is not in anyway attached to the main path commencing in $R_1$ will not be reachable includding those conected to the main path but with no transition towards them.

    To end the computation of $M$'s 5-tuple we shall find the accept states $F'$. By definition, the accept states are $F' = \{R \in Q' | R \text{ contains an accpet state of } N\}$. Thus,
    \begin{equation*}
        F' = \{R_{4}, R_{14}, R_{24}, R_{34}, R_{124}, R_{134}, R_{234}, R_{1234}\}\ .
    \end{equation*}

    The resulting DFA $M$ from the transition table is shown in \cref{fig:det-automaton-general}. Clearly, the states $R_{1234}$ and $R_{123}$ are both connected to the main path but have no connection towards them, these two states are unreachable. Furthermore, there is a set of unreachable states which do not have any connection to the main path nor a start state.
    \begin{figure}[ht!]
        \centering
        \includestandalone[mode=buildnew]{det-automaton-path1}
        \includestandalone[mode=buildnew]{det-automaton-path2}
        \caption{State diagram of the DFA $M$. Main path shown in the left, and unreachable set of states shown in the right.}
        \label{fig:det-automaton-general}
    \end{figure}

    Therefore, after purging the states that cannot be reached, the automaton $M$ is left as the form shown in \cref{fig:det-automaton}, and it is the DFA equivalent to the NFA $N$.
    \begin{figure}[ht!]
        \centering
        \includestandalone[mode=buildnew]{det-automaton}
        \caption{Final state diagram of $M$.}
        \label{fig:det-automaton}
    \end{figure}
    
\end{solution}


\end{document}