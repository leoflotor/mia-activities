% -- Document configuration
\documentclass{article}

% -- Input and language settings
% \usepackage[utf8]{inputenc}
\usepackage[spanish]{babel}
\decimalpoint                             % From babel package to use points instead of commas in decimals

% -- Page and line settings
\usepackage{geometry}
\geometry{letterpaper, 
    % margin=2cm, 
    left=3cm, right=3cm,
    top=1.2cm, bottom=1.2cm,
    includefoot, 
    includehead}
\renewcommand{\baselinestretch}{1.2}

% -- Required packages
\usepackage{xcolor}
\usepackage[many]{tcolorbox}
\usepackage{mathtools,amsfonts,amsmath}     % Loads amsmath if not already loaded
\allowdisplaybreaks                         % To allow page breaks if equations are too long
\usepackage[parfill]{parskip}               % No indent and separation lines for paragraphs
\usepackage{cancel}                         % To cancel math terms
\usepackage[shortlabels]{enumitem}          % To handle enumerations
\usepackage{tikz}
\usetikzlibrary{automata, arrows.meta, positioning}
\usepackage[mode=buildnew]{standalone}      % To import figures in standalone files
\usepackage[hidelinks]{hyperref}
\usepackage[spanish]{cleveref}              % To use autocompleted reference labels, language must be change as in babel package
\usepackage{caption}                        % Caption and subcaption to allow subfigures
\usepackage{subcaption}
\usepackage{float}                          % To specify the location of figures
\usepackage{multicol}                       % To use multicolumns
\usepackage[bottom]{footmisc}               % To locate footnotes at the bottom

% -- Title and heading settings
\usepackage{titling}
\usepackage{fancyhdr}
\pagestyle{fancy}

% -- Code and code formatting
\usepackage{minted}                         % To insert code
\usemintedstyle[julia]{gruvbox-light}       % Code theme and language
\definecolor{bg}{rgb}{0.98, 0.97, 0.88}     % Code block background

\usepackage{fontspec}                       % To allow the use of monospace fonts
\setmonofont{JuliaMono}[Path=./codefonts/, Extension=.ttf, UprightFont=*-Regular, ItalicFont=*-RegularItalic, Scale=0.75]

\usepackage{fancyvrb}                       % To change line number font
\renewcommand{\theFancyVerbLine}{\textcolor{gray}{\footnotesize\texttt{\arabic{FancyVerbLine}}}}

\definecolor{light-gray}{gray}{0.95}        % Color, box and style to show small code thingys inside normal text
\newcommand{\code}[1]{\colorbox{light-gray}{\texttt{#1}}}

% -- Bilbiography preferences
\usepackage[square,numbers]{natbib}
\bibliographystyle{unsrt}

% -- Footnotes without numbering
\newcommand\nnfootnote[1]{%
  \begin{NoHyper}
  \renewcommand\thefootnote{}\footnote{#1}%
  \addtocounter{footnote}{-1}%
  \end{NoHyper}
}

% -- Theorems
\newtheorem{theorem}{Theorem}

\lhead{\theinstitution\ -- \thedepartment}
\chead{}
\rhead{Programación para la IA\ -- \thetitle}
\lfoot{}
\cfoot{\thepage}
\rfoot{}

% -- Problem solution
\newenvironment{solution}
{\begin{quote}
\textbf{Solución:}\medskip

}
{

\hfill\rule{0.5\textwidth}{0.5pt}
\end{quote}}

% -- Equation result
\newcommand{\result}[1]
{
\tcbhighmath[colframe=white, colback=gray!15, sharp corners]
{#1}
}

% -- Function definitions
\newcommand{\dprod}[2]{{#1} \cdot {#2}}
\newcommand{\txtgray}[1]{\textcolor{gray}{#1}}

% -- Author information
\title{Actividad 5}
\author{Leonardo Flores Torres}
\newcommand\theinstitution{Universidad Veracruzana}
\newcommand\thedepartment{Inteligencia Artificial}
\newcommand\thecourse{Programación para la Inteligencia Artificial}

% -- Paths
% \newcommand\codelists{../programs/lists.rkt}

% Remove red color boxes of "syntax errors" in minted
\AtBeginEnvironment{minted}{%
  \renewcommand{\fcolorbox}[4][]{#4}}

% -- Document
\begin{document}

\thispagestyle{empty}

%Title
\begin{center}
\textsc{\theinstitution}\\[2mm]

\thedepartment

\rule{0.6\textwidth}{0.5pt}\\[2mm]

\thecourse \\[4mm]

{\Large \textbf{\thetitle}}\\[2mm]

\theauthor \\[2mm]

{\small \today}
\end{center}
\medskip

% -- 
\vspace{1cm}

La gramática libre de contexto a utilizar para este proyecto es
\begin{equation}
    \begin{split}
        \text{S} & \to \text{xSz} \\
        \text{S} & \to \text{xyTyz} \\
        \text{T} & \to \text{\lambda}
    \end{split}
    \label{grammar_1}
\end{equation}
pero también se quiso tomar a la gramática libre de contexto vista en clase
\begin{equation}
    \begin{split}
        \text{S} & \to \text{zMNz} \\
        \text{M} & \to \text{aMa} \\
        \text{M} & \to \text{z} \\
        \text{N} & \to \text{bNb} \\
        \text{N} & \to \text{z}
    \end{split}
    \label{grammar_2}
\end{equation}
para corroborar los resultados del autómata obtenido, la tabla del parser, y para probar la generalidad de la implementación del algoritmo escrito en \code{julia} \cite{Bezanson_Julia_A_fresh_2017}. La implementación se incluye en el Apéndice C de este trabajo.

\begin{figure}[ht!]
    \centering
    \begin{tikzpicture}[draw=gray!70!black, node distance = 2cm, on grid, auto, every loop/.style={stealth-}]
        \node (r1) [state, initial, initial text = {}] {$1.1$};
        \node (r2) [state, right = of r1] {$1.2$};
        \node (r3) [state, below left = of r2] {$1.3$};
        \node (r4) [state, below right = of r2] {$1.4$};
        \node (r5) [state, below left = of r3, accepting] {$1.5$};
        \node (r6) [state, above right = of r4, accepting] {$1.6$};
        \node (r7) [state, below right = of r4] {$1.7$};
        \node (r8) [state, right = of r7] {$1.8$};
        \node (r9) [state, right = of r8, accepting] {$1.9$};
        \node (r10) [state, above = of r1, accepting] {$1.10$};

        \path [-stealth, thick]
            (r1) edge node {\textcolor{gray}{x}} (r2)
            (r1) edge node {\textcolor{gray}{S}} (r10)
            (r2) edge node {\textcolor{gray}{S}} (r3)
            (r2) edge node {\textcolor{gray}{y}} (r4)
            (r2) edge[loop above] node {\textcolor{gray}{x}} ()
            (r3) edge node {\textcolor{gray}{z}} (r5)
            (r4) edge node {\textcolor{gray}{$\lambda$}} (r6)
            (r4) edge node {\textcolor{gray}{T}} (r7)
            (r7) edge node {\textcolor{gray}{y}} (r8)
            (r8) edge node {\textcolor{gray}{S}} (r9);
    \end{tikzpicture}
    \caption{Autómata finito para la gramática 1.}
    \label{fig:automata_grammar_1}
\end{figure}

Los estados del autómata de la \cref{grammar_1} se incluyen en el Apéndice A, y los del autómata correspondiente a la \cref{grammar_2} en el Apéndice B, en ambos casos los estados aparecen enumerados para coincidir con los diagramas de sus respectivos autómatas finitos. De igual manera, los autómatas finitos de las \cref{grammar_1,grammar_2} se muestran en las \cref{fig:automata_grammar_1,fig:automata_grammar_2}, y sus respectivas tablas de parsers en las \cref{table:grammar_1,table:grammar_2} al final del documento.

\begin{figure}[ht!]
    \centering
    \begin{tikzpicture}[draw=gray!70!black, node distance = 2cm, on grid, auto, every loop/.style={stealth-}]
        \node (r1) [state, initial, initial text = {}] {$2.1$};
        \node (r2) [state, below = of r1] {$2.2$};
        \node (r3) [state, below = of r2] {$2.3$};
        \node (r4) [state, above right = of r2] {$2.4$};
        \node (r5) [state, right = of r4] {$2.5$};
        \node (r6) [state, above right = of r4] {$2.6$};
        \node (r7) [state, below left = of r2, accepting] {$2.7$};
        \node (r8) [state, right = of r3] {$2.8$};
        \node (r9) [state, below = of r4, accepting] {$2.9$};
        \node (r10) [state, right = of r5] {$2.10$};
        \node (r11) [state, right = of r6, accepting] {$2.11$};
        \node (r12) [state, right = of r8, accepting] {$2.12$};
        \node (r13) [state, below = of r10, accepting] {$2.13$};
        \node (r14) [state, above right = of r1, accepting] {$2.14$};

        \path [-stealth, thick] 
            (r1) edge node[left] {\textcolor{gray}{z}} (r2)
            (r1) edge node {\textcolor{gray}{S}} (r14)
            (r2) edge node {\textcolor{gray}{M}} (r4)
            (r2) edge node {\textcolor{gray}{a}} (r3)
            (r2) edge node {\textcolor{gray}{z}} (r7)
            (r3) edge[loop below] node {\textcolor{gray}{a}} ()
            (r3) edge node {\textcolor{gray}{M}} (r8)
            (r3) edge node {\textcolor{gray}{z}} (r7)
            (r4) edge node {\textcolor{gray}{N}} (r6)
            (r4) edge node {\textcolor{gray}{b}} (r5)
            (r4) edge node {\textcolor{gray}{z}} (r9)
            (r5) edge[loop below] node {\textcolor{gray}{b}} ()
            (r5) edge node {\textcolor{gray}{N}} (r10)
            (r5) edge node {\textcolor{gray}{z}} (r9)
            (r6) edge node {\textcolor{gray}{z}} (r11)
            (r8) edge node {\textcolor{gray}{a}} (r12)
            (r10) edge node {\textcolor{gray}{b}} (r13);
    \end{tikzpicture}
    \caption{Autómata finito para la gramática 2.}
    \label{fig:automata_grammar_2}
\end{figure}

Se puede utilizar a la \cref{grammar_2} para encontrar algunas cadenas, por ejemplo:
\begin{equation*}
    \begin{split}
        \text{S} & \hspace{1cm} \rewrule{S}{zMNz} \\
        \text{zMNz} & \hspace{1cm} \rewrule{N}{bNb} \\
        \text{zMbNbz} & \hspace{1cm} \rewrule{N}{z} \\
        \text{zMbzbz} & \hspace{1cm} \rewrule{M}{aMa} \\
        \text{zaMabzbz} & \hspace{1cm} \rewrule{M}{z} \\
        \text{zazabzbz} & 
    \end{split}
\end{equation*}
La siguiente cadena también es válida para esta gramática:
\begin{equation*}
    \begin{split}
        \text{S} & \hspace{1cm} \rewrule{S}{zMNz} \\
        \text{zMNz} & \hspace{1cm} \rewrule{N}{bNb} \\
        \text{zMbNbz} & \hspace{1cm} \rewrule{N}{bNb} \\
        \text{zMbbNbbz} & \hspace{1cm} \rewrule{N}{z} \\
        \text{zMbbzbbz} & \hspace{1cm} \rewrule{M}{z} \\
        \text{zzbbzbbz} & 
    \end{split}
\end{equation*}
Lo interesante ahora sería probar estas cadenas con la implementación del algoritmo. Primero se carga el módulo hecho en \code{julia}, se define la cadena de entrada en la variable \code{input\_string} y se llama a la rutina del algoritmo llamando a la función \code{grammar\_two} donde se encuentra definida a la \cref{grammar_2} y al final de esta función se llama internamente a la función \code{parse\_string}. Lo anteriormente mencionado se muestra a continuación:
\begin{minted}[
    frame=none,
    autogobble,
    obeytabs=false,
    breaklines,
    tabsize=4,
    linenos=true,
    baselinestretch=1,
    firstnumber=last,
    bgcolor=bg!70,
    mathescape,
    numberblanklines=false
    ]{julia}
    julia> using Revise; using LR1Parser; lrp = LR1Parser;

    julia> input_string = "zazabzbz"; lrp.grammar_two(input_string)    # Primera cadena de prueba
    Iteration 1: Any[1] 

    Current symbol: z 

    Iteration 2: Any[1, "z", 2] 

    Current symbol: a 

    Iteration 3: Any[1, "z", 2, "a", 3] 

    Current symbol: z 

    Iteration 4: Any[1, "z", 2, "a", 3, "z", 7] 

    Current symbol: a 

    Iteration 5: Any[1, "z", 2, "a", 3, "M", 8] 

    Iteration 6: Any[1, "z", 2, "a", 3, "M", 8, "a", 12] 

    Current symbol: b 

    Iteration 7: Any[1, "z", 2, "M", 4] 

    Iteration 8: Any[1, "z", 2, "M", 4, "b", 5] 

    Current symbol: z 

    Iteration 9: Any[1, "z", 2, "M", 4, "b", 5, "z", 9] 

    Current symbol: b 

    Iteration 10: Any[1, "z", 2, "M", 4, "b", 5, "N", 10] 

    Iteration 11: Any[1, "z", 2, "M", 4, "b", 5, "N", 10, "b", 13] 

    Current symbol: z 

    Iteration 12: Any[1, "z", 2, "M", 4, "N", 6] 

    Iteration 13: Any[1, "z", 2, "M", 4, "N", 6, "z", 11] 

    Current symbol: eos 

    Iteration 14: Any[1, "S", 14] 

    The string "zazabzbz" was succesfully parsed.

    julia> input_string = "zzbbzbbz"; lrp.grammar_two(input_string)    # Segunda cadena de prueba
    Iteration 1: Any[1] 

    Current symbol: z 

    Iteration 2: Any[1, "z", 2] 

    Current symbol: z 

    Iteration 3: Any[1, "z", 2, "z", 7] 

    Current symbol: b 

    Iteration 4: Any[1, "z", 2, "M", 4] 

    Iteration 5: Any[1, "z", 2, "M", 4, "b", 5] 

    Current symbol: b 

    Iteration 6: Any[1, "z", 2, "M", 4, "b", 5, "b", 5] 

    Current symbol: z 

    Iteration 7: Any[1, "z", 2, "M", 4, "b", 5, "b", 5, "z", 9] 

    Current symbol: b 

    Iteration 8: Any[1, "z", 2, "M", 4, "b", 5, "b", 5, "N", 10] 

    Iteration 9: Any[1, "z", 2, "M", 4, "b", 5, "b", 5, "N", 10, "b", 13] 

    Current symbol: b 

    Iteration 10: Any[1, "z", 2, "M", 4, "b", 5, "N", 10] 

    Iteration 11: Any[1, "z", 2, "M", 4, "b", 5, "N", 10, "b", 13] 

    Current symbol: z 

    Iteration 12: Any[1, "z", 2, "M", 4, "N", 6] 

    Iteration 13: Any[1, "z", 2, "M", 4, "N", 6, "z", 11] 

    Current symbol: eos 

    Iteration 14: Any[1, "S", 14] 

    The string "zzbbzbbz" was succesfully parsed.
\end{minted}

Se pensó que sería útil mostrar los pasos de cada iteración del \code{while-loop} dentro de la implementación del algoritmo, de esta manera se podría corroborar el estado del \code{stack} conforme se agregan símbolos y los números correspondientes a los estados en el autómata finito definidos como \code{token}s. Además, para cadenas cortas, esto serviría para seguir los pasos a mano y realizar una comparación mas detalladamente de los contenidos del stack cada que se realiza un \code{pop} o un \code{push}.

Lo mismo se puede hacer para la \cref{grammar_1}. Una cadena que se puede deducir a partir de esta gramática es la siguiente:
\begin{equation*}
    \begin{split}
        \text{S} & \hspace{1cm} \rewrule{S}{xSz} \\
        \text{xSz} & \hspace{1cm} \rewrule{S}{xyTyz} \\
        \text{xxyTyzz} & \hspace{1cm} \rewrule{T}{$\lambda$} \\
        \text{xxyyzz} & 
    \end{split}
\end{equation*}
Una segunda cadena perteneciente a esta gramática es:
\begin{equation*}
    \begin{split}
        \text{S} & \hspace{1cm} \rewrule{S}{xSz} \\
        \text{xSz} & \hspace{1cm} \rewrule{S}{xSz} \\
        \text{xxSzz} & \hspace{1cm} \rewrule{S}{xyTyz} \\
        \text{xxxyTyzzz} & \hspace{1cm} \rewrule{T}{$\lambda$} \\
        \text{xxxyyzzz} & 
    \end{split}
\end{equation*}
Con estas dos cadenas se probará el algoritmo. De manera similar al caso anterior, se definió una función \code{grammar\_one} en donde se define a la \cref{grammar_1}, e internamente se llama a \code{parse\_string}:
\begin{minted}[
    frame=none,
    autogobble,
    obeytabs=false,
    breaklines,
    tabsize=4,
    linenos=true,
    baselinestretch=1,
    firstnumber=last,
    bgcolor=bg!70,
    mathescape,
    numberblanklines=false
    ]{julia}
    julia> input_string = "xxyyzz"; lrp.grammar_one(input_string)    # Primera cadena de prueba
    Iteration 1: Any[1] 

    Current symbol: x 

    Iteration 2: Any[1, "x", 2] 

    Current symbol: x 

    Iteration 3: Any[1, "x", 2, "x", 2] 

    Current symbol: y 

    Iteration 4: Any[1, "x", 2, "x", 2, "y", 4] 

    Current symbol: y 

    Iteration 5: Any[1, "x", 2, "x", 2, "y", 4, "T", 7] 

    Iteration 6: Any[1, "x", 2, "x", 2, "y", 4, "T", 7, "y", 8] 

    Current symbol: z 

    Iteration 7: Any[1, "x", 2, "x", 2, "y", 4, "T", 7, "y", 8, "z", 9] 

    Current symbol: z 

    Iteration 8: Any[1, "x", 2, "S", 3] 

    Iteration 9: Any[1, "x", 2, "S", 3, "z", 5] 

    Current symbol: eos 

    Iteration 10: Any[1, "S", 10] 

    The string "xxyyzz" was succesfully parsed.

    julia> input_string = "xxxyyzzz"; lrp.grammar_one(input_string)    # Segunda cadena de prueba
    Iteration 1: Any[1] 

    Current symbol: x 

    Iteration 2: Any[1, "x", 2] 

    Current symbol: x 

    Iteration 3: Any[1, "x", 2, "x", 2] 

    Current symbol: x 

    Iteration 4: Any[1, "x", 2, "x", 2, "x", 2] 

    Current symbol: y 

    Iteration 5: Any[1, "x", 2, "x", 2, "x", 2, "y", 4] 

    Current symbol: y 

    Iteration 6: Any[1, "x", 2, "x", 2, "x", 2, "y", 4, "T", 7] 

    Iteration 7: Any[1, "x", 2, "x", 2, "x", 2, "y", 4, "T", 7, "y", 8] 

    Current symbol: z 

    Iteration 8: Any[1, "x", 2, "x", 2, "x", 2, "y", 4, "T", 7, "y", 8, "z", 9] 

    Current symbol: z 

    Iteration 9: Any[1, "x", 2, "x", 2, "S", 3] 

    Iteration 10: Any[1, "x", 2, "x", 2, "S", 3, "z", 5] 

    Current symbol: z 

    Iteration 11: Any[1, "x", 2, "S", 3] 

    Iteration 12: Any[1, "x", 2, "S", 3, "z", 5] 

    Current symbol: eos 

    Iteration 13: Any[1, "S", 10] 

    The string "xxxyyzzz" was succesfully parsed.
\end{minted}

Un par de elementos que se añadió al algoritmo en \code{parse\_string} es entrar tempranamente a la rutina de error si detecta que un caracter en la cadena de entrada no pertenece al alfabeto de la gramática. Esto se implementó como se muestra a continuación (se puede observar más detalladamente este extracto de código en el Apéndice C):
\inputminted[
    frame=none,
    autogobble,
    obeytabs=false,
    breaklines,
    tabsize=4,
    linenos=false,
    baselinestretch=1,
    firstnumber=1,
    firstline=35,
    lastline=41,
    bgcolor=bg!70,
    ]{julia}{\codepath}

Por ejemplo, si se usara la siguiente cadena de entrada \code{\quotes{xxybyzz}} con la \cref{grammar_1}, el caracter \code{\quotes{b}} no es parte del alfabeto de esa gramática, y el algoritmo debería entrar a la rutina de error:
\begin{minted}[
    frame=none,
    autogobble,
    obeytabs=false,
    breaklines,
    tabsize=4,
    linenos=true,
    baselinestretch=1,
    firstnumber=95,
    bgcolor=bg!70,
    mathescape,
    numberblanklines=false
    ]{julia}
    julia> input_string = "xxybyzz"; lrp.grammar_one(input_string)
    ERROR: One or more characters in the input string "xxybyzz" do not belong to the symbols ["eos", "S", "T", "x", "z", "y"] of the grammar.
\end{minted}
Recuérdese que a pesar de que se llame a la función \code{grammar\_one}, esta a su vez llama internamente a \code{parse\_string} entonces siempre se está llamando al algoritmo del parser. Otra rutina de error que se añadió fue si se llegara a detectar el caso en que el símbolo al tope del \code{stack}, por algún motivo, no coincide con el símbolo en curso del lado derecho de la regla de reescritura:
\inputminted[
    frame=none,
    autogobble,
    obeytabs=false,
    breaklines,
    tabsize=4,
    linenos=false,
    baselinestretch=1,
    firstnumber=1,
    firstline=73,
    lastline=86,
    bgcolor=bg!70,
    ]{julia}{\codepath}

Esta última rutina de error se añadió al momento de ir probando el buen funcionamiento del algoritmo y se decidió mantenerla. Se mantuvo la última operación de vaciar el \code{stack} al final del algoritmo, aunque en \code{julia} no es necesario ya que al momento de salir de la función \code{parse\_string} toda variable local (variable definida al interior de una función) que no sea regresada por la función misma se limpia.

La implementación solamente se hizo para un parser del tipo $\text{LR}(k=1)$, esto quiere decir que el parser solamente puede leer un símbolo a la vez. Como trabajo posterior sería interesante pensar en cómo realizar la implementación para $k \neq 1$, con esto me refiero a poder elegir un valor arbitrario de $k$ sin que la implementación sea dependiente del valor asignado. El parser hecho en este trabajo no puede manejar otro valor de $k$ que no sea 1. Lo difícil en esto no es el algoritmo en \code{parse\_string} per se sino en idear un algoritmo que genere automáticamente la tabla de transiciones a partir de $k$.

Finalmente, se quisiera incluir un caso extra que resulta en error. El algoritmo puede manejar 3 condiciones dentro del \code{while-loop}. La primera es cuando la entrada en la tabla de su respectiva gramática es un \code{\quotes{shift}}, la segunda condición es cuando la entrada es una regla de reescritura (también llamada reducción) la cual en este trabajo se identificó como \code{\quotes{redux}}, y la tercera condición es cuando la entrada de la tabla está vacía. Este último caso, cuando la entrada está vacía, es el que detona una rutina de error y se dá cuando todos los símbolos de la cadena de entrada sí pertenecen al alfabeto de la gramática pero la cadena no puede ser generada a partir de la misma. Esto se ejemplificará con las cadenas \code{\quotes{xxxyyyzzz}} y \code{\quotes{zzzbbzbbz}} de las \cref{grammar_1,grammar_2}, respectivamente. A continuación se muestra lo descrito anteriormente:
\begin{minted}[
    frame=none,
    autogobble,
    obeytabs=false,
    breaklines,
    tabsize=4,
    linenos=true,
    baselinestretch=1,
    firstnumber=97,
    bgcolor=bg!70,
    mathescape,
    numberblanklines=false
    ]{julia}
    julia> input_string = "xxxyyyzzz"; lrp.grammar_one(input_string)    # Primera prueba de error
    Iteration 1: Any[1] 

    Current symbol: x 

    Iteration 2: Any[1, "x", 2] 

    Current symbol: x 

    Iteration 3: Any[1, "x", 2, "x", 2] 

    Current symbol: x 

    Iteration 4: Any[1, "x", 2, "x", 2, "x", 2] 

    Current symbol: y 

    Iteration 5: Any[1, "x", 2, "x", 2, "x", 2, "y", 4] 

    Current symbol: y 

    Iteration 6: Any[1, "x", 2, "x", 2, "x", 2, "y", 4, "T", 7] 

    Iteration 7: Any[1, "x", 2, "x", 2, "x", 2, "y", 4, "T", 7, "y", 8] 

    Current symbol: y 

    ERROR: Value of current table entry [8, "y"] is missing/blank.

    julia> input_string = "zzzbbzbbz"; lrp.grammar_two(input_string)    # Segunda prueba de error
    Iteration 1: Any[1] 

    Current symbol: z 

    Iteration 2: Any[1, "z", 2] 

    Current symbol: z 

    Iteration 3: Any[1, "z", 2, "z", 7] 

    Current symbol: z 

    Iteration 4: Any[1, "z", 2, "M", 4] 

    Iteration 5: Any[1, "z", 2, "M", 4, "z", 9] 

    Current symbol: b 

    Iteration 6: Any[1, "z", 2, "M", 4, "N", 6] 

    ERROR: Value of current table entry [6, "b"] is missing/blank.
\end{minted}

\begin{table}[ht!]
    \centering
    \setlength{\tabcolsep}{8pt}
    \begin{tabular}{ c|c c c c c c }
        & x & y & z & EOS & S & T \\
        \hline
        1 & shift 2 & & & & 10 & \\
        2 & shift 2 & shift 4 & & & 3 & \\
        3 & & & shift 5 & & & \\
        4 & & \rewrule{T}{$\lambda$} & & & & 7 \\
        5 & & & \rewrule{S}{xSz} & \rewrule{S}{xSz} & & \\
        6 & & \rewrule{T}{$\lambda$} & & & & \\
        7 & & shift 8 & & & & \\
        8 & & & shift 9 & & & \\
        9 & & & \rewrule{S}{xyTyz} & \rewrule{S}{xyTyz} & & \\
        10 & & & & accept & & \\
    \end{tabular}
    \caption{Tabla del parser LR(1) basado en la \cref{grammar_1}.}
    \label{table:grammar_1}
\end{table}

\begin{table}[ht!]
    \centering
    \setlength{\tabcolsep}{8pt}
    \begin{tabular}{ c|c c c c c c c }
        & a & b & z & EOS & S & N & N \\
        \hline
        1 & & & shift 2 & & 14 & & \\
        2 & shift 3 & & shift 7 & & & 4 & \\
        3 & shift 3 & & shift 7 & & & 8 & \\
        4 & & shift 5 & shift 9 & & & & 6 \\
        5 & & shift 5 & shift 9 & & & & 10 \\
        6 & & & shift 11 & & & & \\
        7 & \rewrule{M}{z} & \rewrule{M}{z} & \rewrule{M}{z} & & & & \\
        8 & shift 12 & & & & & & \\
        9 & & \rewrule{N}{z} & \rewrule{N}{z} & & & & \\
        10 & & shift 13 & & & & & \\
        11 & & & & \rewrule{S}{zMNz} & & & \\
        12 & \rewrule{M}{aMa} & \rewrule{M}{aMa} & \rewrule{M}{aMa} & & & & \\
        13 & & \rewrule{N}{bNb} & \rewrule{N}{bNb} & & & & \\
        14 & & & & accept & & & \\
    \end{tabular}
    \caption{Tabla del parser LR(1) basado en la \cref{grammar_2}.}
    \label{table:grammar_2}
\end{table}

Aquí termina el trabajo de la implementación del parser LR(1), el leer y entender el trabajo hecho por Brookshear y Sipser \cite{brookshear1989theory,sipser2021introduction} lo suficiente como para implementar un parser de este tipo me hace preguntarme que cosas tan interesantes estan por venir y que más hay por aprender acerca de la teoría de la computación ¿Qué más hay por aprender sobre autómatas y lenguajes formales? Habrá que averiguarlo.

\clearpage
\section*{Apéndice A\hspace{0.5cm} Estados de la primera gramática}

Estados del autómata finito mostrado en la \cref{fig:automata_grammar_1} generado a partir de la \cref{grammar_1}. El símbolo $\marker$ funge como el marcador de lectura.
\vspace{1cm}

\begin{minipage}{0.45\textwidth}
    \centering
    %
    \begin{tcolorbox}[ams equation, colback=lightgray!15, colframe=gray, width=55mm]
        \begin{split}
            \text{S}' \to & \marker \text{S} \\
            \text{S} \to & \marker \text{xSz} \\
            \text{S} \to & \marker \text{xyTyz}
        \end{split}
        \tag{1.1}
    \end{tcolorbox}
    %
    \begin{tcolorbox}[ams equation, colback=lightgray!15, colframe=gray, width=55mm]
        \begin{split}
            \text{S} \to & \text{x} \marker \text{Sz} \\
            \text{S} \to & \text{x} \marker \text{yTyz} \\
            \text{S} \to & \marker \text{xSz} \\
            \text{S} \to & \marker \text{xyTyz}
        \end{split}
        \tag{1.2}
    \end{tcolorbox}
    %
    \begin{tcolorbox}[ams equation, colback=lightgray!15, colframe=gray, width=55mm]
        \begin{split}
            \text{S} \to & \text{xS} \marker \text{z}
        \end{split}
        \tag{1.3}
    \end{tcolorbox}
    %
    \begin{tcolorbox}[ams equation, colback=lightgray!15, colframe=gray, width=55mm]
        \begin{split}
            \text{S} \to & \text{xy} \marker \text{Tyz} \\
            \text{T} \to & \marker \text{\lambda}
        \end{split}
        \tag{1.4}
    \end{tcolorbox}
    %
    \begin{tcolorbox}[ams equation, colback=lightgray!15, colframe=gray, width=55mm]
        \begin{split}
            \text{S} \to & \text{xSz} \marker
        \end{split}
        \tag{1.5}
    \end{tcolorbox}
\end{minipage}
\begin{minipage}{0.45\textwidth}
    \centering
    %
    \begin{tcolorbox}[ams equation, colback=lightgray!15, colframe=gray, width=55mm]
        \begin{split}
            \text{T} \to & \text{\lambda} \marker
        \end{split}
        \tag{1.6}
    \end{tcolorbox}
    %
    \begin{tcolorbox}[ams equation, colback=lightgray!15, colframe=gray, width=55mm]
        \begin{split}
            \text{S} \to & \text{xyT} \marker \text{yz}
        \end{split}
        \tag{1.7}
    \end{tcolorbox}
    %
    \begin{tcolorbox}[ams equation, colback=lightgray!15, colframe=gray, width=55mm]
        \begin{split}
            \text{S} \to & \text{xyTy} \marker \text{z}
        \end{split}
        \tag{1.8}
    \end{tcolorbox}
    %
    \begin{tcolorbox}[ams equation, colback=lightgray!15, colframe=gray, width=55mm]
        \begin{split}
            \text{S} \to & \text{xyTyz} \marker
        \end{split}
        \tag{1.9}
    \end{tcolorbox}
    %
    \begin{tcolorbox}[ams equation, colback=lightgray!15, colframe=gray, width=55mm]
        \begin{split}
            \text{S}' \to & \text{S} \marker
        \end{split}
        \tag{1.10}
    \end{tcolorbox}
\end{minipage}

\clearpage
\section*{Apéndice B\hspace{0.5cm} Estados de la segunda gramática}

Estados del autómata finito mostrado en la \cref{fig:automata_grammar_2} generado a partir de la \cref{grammar_2}. El símbolo $\marker$ funge como el marcador de lectura.
\vspace{1cm}

\begin{minipage}{0.45\textwidth}
    \centering
    %
    \begin{tcolorbox}[ams equation, colback=lightgray!15, colframe=gray, width=55mm]
        \begin{split}
            \text{S}' \to & \marker \text{S} \\
            \text{S} \to & \marker \text{zMNz}
        \end{split}
        \tag{2.1}
    \end{tcolorbox}
    %
    \begin{tcolorbox}[ams equation, colback=lightgray!15, colframe=gray, width=55mm]
        \begin{split}
            \text{S} \to & \text{z} \marker \text{MNz} \\
            \text{M} \to & \marker \text{aMa} \\
            \text{M} \to & \marker \text{z}
        \end{split}
        \tag{2.2}
    \end{tcolorbox}
    %
    \begin{tcolorbox}[ams equation, colback=lightgray!15, colframe=gray, width=55mm]
        \begin{split}
            \text{M} \to & \text{a} \marker \text{Ma} \\
            \text{M} \to & \marker \text{aMa} \\
            \text{M} \to & \marker \text{z}
        \end{split}
        \tag{2.3}
    \end{tcolorbox}
    %
    \begin{tcolorbox}[ams equation, colback=lightgray!15, colframe=gray, width=55mm]
        \begin{split}
            \text{S} \to & \text{zM} \marker \text{Nz} \\
            \text{N} \to & \marker \text{bNb} \\
            \text{N} \to & \marker \text{z}
        \end{split}
        \tag{2.4}
    \end{tcolorbox}
    %
    \begin{tcolorbox}[ams equation, colback=lightgray!15, colframe=gray, width=55mm]
        \begin{split}
            \text{N} \to & \text{b} \marker \text{Nb} \\
            \text{N} \to & \marker \text{bNb} \\
            \text{N} \to & \marker \text{z}
        \end{split}
        \tag{2.5}
    \end{tcolorbox}
    %
    \begin{tcolorbox}[ams equation, colback=lightgray!15, colframe=gray, width=55mm]
        \begin{split}
            \text{S} \to & \text{zMN} \marker \text{z}
        \end{split}
        \tag{2.6}
    \end{tcolorbox}
\end{minipage}
\begin{minipage}{0.45\textwidth}
    \centering
    %
    \begin{tcolorbox}[ams equation, colback=lightgray!15, colframe=gray, width=55mm]
        \begin{split}
            \text{M} \to & \text{z} \marker
        \end{split}
        \tag{2.7}
    \end{tcolorbox}
    %
    \begin{tcolorbox}[ams equation, colback=lightgray!15, colframe=gray, width=55mm]
        \begin{split}
            \text{M} \to & \text{aM} \marker \text{a}
        \end{split}
        \tag{2.8}
    \end{tcolorbox}
    %
    \begin{tcolorbox}[ams equation, colback=lightgray!15, colframe=gray, width=55mm]
        \begin{split}
            \text{N} \to & \text{z} \marker
        \end{split}
        \tag{2.9}
    \end{tcolorbox}
    %
    \begin{tcolorbox}[ams equation, colback=lightgray!15, colframe=gray, width=55mm]
        \begin{split}
            \text{N} \to & \text{bN} \marker \text{b}
        \end{split}
        \tag{2.10}
    \end{tcolorbox}
    %
    \begin{tcolorbox}[ams equation, colback=lightgray!15, colframe=gray, width=55mm]
        \begin{split}
            \text{S} \to & \text{zMNz} \marker
        \end{split}
        \tag{2.11}
    \end{tcolorbox}
    %
    \begin{tcolorbox}[ams equation, colback=lightgray!15, colframe=gray, width=55mm]
        \begin{split}
            \text{M} \to & \text{aMa} \marker
        \end{split}
        \tag{2.12}
    \end{tcolorbox}
    %
    \begin{tcolorbox}[ams equation, colback=lightgray!15, colframe=gray, width=55mm]
        \begin{split}
            \text{N} \to & \text{bNb} \marker
        \end{split}
        \tag{2.13}
    \end{tcolorbox}
    %
    \begin{tcolorbox}[ams equation, colback=lightgray!15, colframe=gray, width=55mm]
        \begin{split}
            \text{S}' \to & \text{S} \marker
        \end{split}
        \tag{2.14}
    \end{tcolorbox}
\end{minipage}

\clearpage
\section*{Apéndice C\hspace{0.5cm} Implementación del algoritmo}
\vspace{1cm}
\inputminted[
    frame=none,
    autogobble,
    obeytabs=false,
    breaklines,
    tabsize=4,
    linenos=true,
    baselinestretch=1,
    firstnumber=1,
    bgcolor=bg!70,]{julia}{\codepath}

\nocite{*}    % to call all references even if they are not cited in the text
\bibliography{references.bib}

\end{document}

% \begin{minted}[
%     frame=none,
%     autogobble,
%     obeytabs=false,
%     breaklines,
%     tabsize=4,
%     linenos=true,
%     baselinestretch=1,
%     firstnumber=last,
%     bgcolor=bg!70,
%     mathescape,
%     numberblanklines=false
%     ]{julia}
%     # empty comment
% \end{minted}


