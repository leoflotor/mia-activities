% -- Document configuration
\documentclass{article}

% -- Input and language settings
% \usepackage[utf8]{inputenc}
\usepackage[spanish]{babel}
\decimalpoint                             % From bable package to use points instead of commas in decimals

% -- Page and line settings
\usepackage{geometry}
\geometry{letterpaper, 
    % margin=2cm, 
    left=2cm, right=2cm,
    top=1.2cm, bottom=1.2cm,
    includefoot, 
    includehead}
\renewcommand{\baselinestretch}{1.2}

% -- Required packages
\usepackage{xcolor}
\usepackage[many]{tcolorbox}
\usepackage{blkarray}                       % Allow matrices with row and column labels
\usepackage{mathtools,amsfonts,amsmath}     % Loads amsmath if not already loaded
\allowdisplaybreaks                         % To allow page breaks if equations are too long
\usepackage[parfill]{parskip}               % No indent and separation lines for paragraphs
\usepackage{cancel}                         % To cancel math terms
\usepackage[shortlabels]{enumitem}          % To handle enumerations
\usepackage{tikz}
\usetikzlibrary{plotmarks}
\usetikzlibrary{math}
\usepackage{pgfplots}
\pgfplotsset{compat=newest}
\usepackage[mode=buildnew]{standalone}      % To import figures in standalone files
\usepackage[hidelinks]{hyperref}
\usepackage[spanish]{cleveref}              % To use autocompleted reference labels, language must be change as in babel package
\usepackage{caption}                        % Caption and subcaption to allow subfigures
\usepackage{subcaption}
\usepackage{float}                          % To specify the location of figures
\usepackage{multicol}                       % To use multicolumns
\usepackage[bottom]{footmisc}               % To locate footnotes at the bottom

% -- Title and heading settings
\usepackage{titling}
\usepackage{fancyhdr}
\pagestyle{fancy}

% -- Code and code formatting
\usepackage{minted}                         % To insert code
\usemintedstyle[julia]{gruvbox-light}       % Code theme and language
\definecolor{bg}{rgb}{0.98, 0.97, 0.88}     % Code block background

\usepackage{textcomp}                       % To allow straight quotes
\usepackage{fontspec}                       % To allow the use of monospace fonts
\setmonofont{JuliaMono}[Path=./codefonts/, Extension=.ttf, UprightFont=*-Regular, ItalicFont=*-RegularItalic, Scale=0.75]

\usepackage{fancyvrb}                       % To change line number font
\renewcommand{\theFancyVerbLine}{\textcolor{gray}{\footnotesize\texttt{\arabic{FancyVerbLine}}}}

\definecolor{light-gray}{gray}{0.95}        % Color, box and style to show small code thingys inside normal text
\newcommand{\code}[1]{\colorbox{light-gray}{\texttt{#1}}}

% -- Bilbiography preferences
\usepackage[square,numbers]{natbib}
\bibliographystyle{unsrt}

% -- Footnotes without numbering
\newcommand\nnfootnote[1]{%
  \begin{NoHyper}
  \renewcommand\thefootnote{}\footnote{#1}%
  \addtocounter{footnote}{-1}%
  \end{NoHyper}
}

% -- Theorems
\newtheorem{theorem}{Theorem}

\lhead{\theinstitution\ -- \thedepartment}
\chead{}
\rhead{\thecourse\ -- \thetitle}
\lfoot{}
\cfoot{\thepage}
\rfoot{}

% -- Problem solution
\newenvironment{solution}
{\begin{quote}
\textbf{Solución:}\medskip

}
{

% \hfill\rule{0.5\textwidth}{0.5pt}
\end{quote}}

% -- Equation result
\newcommand{\result}[1]
{
\tcbhighmath[colframe=white, colback=gray!15, sharp corners]
{#1}
}

% -- Function definitions
\newcommand{\dprod}[2]{{#1} \cdot {#2}}
\newcommand{\txtgray}[1]{\textcolor{gray}{#1}}
\newcommand{\quotes}[1]{\textquotedbl#1\textquotedbl}

% -- Author information
\title{Actividad 6}
\author{Leonardo Flores Torres}
\newcommand\theinstitution{Universidad Veracruzana}
\newcommand\thedepartment{Inteligencia Artificial}
\newcommand\thecourse{Análisis de Algoritmos}

% -- Paths
\newcommand\codepath{../MiniFinder/src/MiniFinder.jl}


% -- Document
\begin{document}

\thispagestyle{empty}

%Title
\begin{center}
\textsc{\theinstitution}\\[2mm]

\thedepartment

\rule{0.6\textwidth}{0.5pt}\\[2mm]

\thecourse \\[4mm]

{\Large \textbf{\thetitle}}\\[2mm]

\theauthor \\[2mm]

{\small \today}
\end{center}
\medskip

% -- 
\vspace{1cm}

\begin{enumerate}
    \item Programe en Prolog un algoritmo de unificación. A reportar \textbf{(50 puntos)}:
    \begin{enumerate}
        \item el algoritmo elegido comentado;
        \item su código, también comentado;
        \item Los siguientes ejemplos de la ejecución:
        \begin{enumerate}
            \item $q(Y, g(a, b))$, $p(g(X, X), Y)$.
            \item $r(a, b, c)$, $r(X, Y, Z)$.
            \item $mayor(padre(Y), Y)$, $mayor(padre(Z), juan)$.
            \item $conoce(padre(X), X)$, $conoce(W, W)$.
        \end{enumerate}
    \end{enumerate}
    %
    \item Implemente las siguientes operaciones sobre conjuntos representados como listas \textbf{(20 puntos)}:
    \begin{itemize}
        \item Subconjunto:
        \begin{minted}[
            frame=none,
            autogobble,
            obeytabs=false,
            breaklines,
            tabsize=4,
            linenos=true,
            % numbersep=-10pt,
            baselinestretch=1,
            firstnumber=1,
            bgcolor=bg!70,
            ]{prolog}
            ?- subset([1,3], [1,2,3,4]).
            true.
            ?- subset([], [1,2]).
            true.
        \end{minted}
        %
        \item Intersección:
        \begin{minted}[
            frame=none,
            autogobble,
            obeytabs=false,
            breaklines,
            tabsize=4,
            linenos=true,
            % numbersep=-10pt,
            baselinestretch=1,
            firstnumber=1,
            bgcolor=bg!70,
            ]{prolog}
            ?- inter([1,2,3], [2,3,4], L).
            L = [2, 3].
        \end{minted}
        %
        \item Unión:
        \begin{minted}[
            frame=none,
            autogobble,
            obeytabs=false,
            breaklines,
            tabsize=4,
            linenos=true,
            % numbersep=-10pt,
            baselinestretch=1,
            firstnumber=1,
            bgcolor=bg!70,
            ]{prolog}
            ?- union([1,2,3,4], [2,3,4,5], L).
            L = [1, 2, 3, 4, 5].
        \end{minted}
        %
        \item Diferencia:
        \begin{minted}[
            frame=none,
            autogobble,
            obeytabs=false,
            breaklines,
            tabsize=4,
            linenos=true,
            % numbersep=-10pt,
            baselinestretch=1,
            firstnumber=1,
            bgcolor=bg!70,
            ]{prolog}
            ?- dif([1,2,3,4], [2,3,4,5], L).
            L = [1].
            ?- dif([1,2,3], [1,4,5], L).
            L = [2, 3].
        \end{minted}
    \end{itemize}
    %
    \item Escriba un predicado que convierta números naturales de Peano a su equivalente decimal. Posteriormente implemente la suma y la resta entre dos números de Peano \textbf{(10 puntos)}. Por ejemplo:
    \begin{minted}[
        frame=none,
        autogobble,
        obeytabs=false,
        breaklines,
        tabsize=4,
        linenos=true,
        % numbersep=-10pt,
        baselinestretch=1,
        firstnumber=1,
        bgcolor=bg!70,
        ]{prolog}
        ?- peanoToNat(s(s(s(0))), N).
        N = 3.
        ?- peanoToNat(0, N).
        N = 0.
        ?- sumaPeano(s(s(0)), s(0), R).
        R = s(s(s(0))).
        ?- restaPeano(s(s(0)), s(0), R).
        R = s(0).
    \end{minted}
    \begin{solution}
        Primero definí una manera de convertir números naturales en la representación de Peano:
        \begin{minted}[
            frame=none,
            autogobble,
            obeytabs=false,
            breaklines,
            tabsize=4,
            linenos=true,
            % numbersep=-10pt,
            baselinestretch=1,
            firstnumber=1,
            bgcolor=bg!70,
            ]{prolog}
            %%% natToPeano/2 Computes the conversion of natural numbers to their Peano representation
            %%% Ejemplo:
            % ?- natToPeano(3, X).
            % X = s(s(s(0))) .
            natToPeano(0, 0).
            natToPeano(N, s(X)) :-
                N > 0,
                N1 is N - 1,
                natToPeano(N1, X).
        \end{minted}
        Con la regla \code{natToPeano/2} traté de hacer otra regla \code{peanoToNat/2} para encontrar el número natural correspondiente a un número de Peano pero no obtenia resultados satisfactorios:
        \begin{minted}[
            frame=none,
            autogobble,
            obeytabs=false,
            breaklines,
            tabsize=4,
            linenos=true,
            % numbersep=-10pt,
            baselinestretch=1,
            firstnumber=1,
            bgcolor=bg!70,
            ]{prolog}
            peanoToNat(P, N) :-
                natToPeano(N, P).
        \end{minted}
        Por lo que definí una regla distinta para este inciso:
        \begin{minted}[
            frame=none,
            autogobble,
            obeytabs=false,
            breaklines,
            tabsize=4,
            linenos=true,
            % numbersep=-10pt,
            baselinestretch=1,
            firstnumber=1,
            bgcolor=bg!70,
            ]{prolog}
            %%% peanoToNat/2 computes the respective natural number of a Peano number
            %%% Example:
            % ?- peanoToNat(s(s(s(0))), X).
            % X = 3.
            peanoToNat(0, 0).
            peanoToNat(s(X), N) :-
                peanoToNat(X, N1),
                N is N1 + 1.
        \end{minted}
        Recordando un poco acerca de los comentarios hechos en clase sobre la librería de dominios finitos decidí utilizarla solo para hacer una definición alternativa de las dos reglas anteriormente definidas:
        \begin{minted}[
            frame=none,
            autogobble,
            obeytabs=false,
            breaklines,
            tabsize=4,
            linenos=true,
            % numbersep=-10pt,
            baselinestretch=1,
            firstnumber=1,
            bgcolor=bg!70,
            ]{prolog}
            %%% With finite domains
            %%% natToPeano_fd/2 computes the conversion of a natural number to its Peano representation
            natToPeano_fd(0, 0).
            natToPeano_fd(N, s(X)) :-
                N #> 0,
                N1 #= N - 1,
                natToPeano_fd(N1, X).
            % peanoToNat_fd/2 computes the conversion of a Peano number to its natural number equivalent
            peanoToNat_fd(S, N) :-
                natToPeano_fd(N, S).
        \end{minted}
        De esta manera sí puedo utilizar \code{natToPeano\_fd/2} para definir \code{peanoToNat\_fd/2}. Cabe mencionar que el resto de las operaciones requeridad para este inciso se elaboraron tomando las reglas que no utilizan la librería de dominio finito.

        La operación de suma se implementó con la siguiente regla:
        \begin{minted}[
            frame=none,
            autogobble,
            obeytabs=false,
            breaklines,
            tabsize=4,
            linenos=true,
            % numbersep=-10pt,
            baselinestretch=1,
            firstnumber=1,
            bgcolor=bg!70,
            ]{prolog}
            %%% addPeano/3 adds S2 to S1, S3 = S1 + S2
            addPeano(S1, S2, S3) :-
                peanoToNat(S1, N1),
                peanoToNat(S2, N2),
                N3 is N1 + N2,
                natToPeano(N3, S3).
        \end{minted}
        De manera similar, la operación de resta entre dos números de Peano se muestra a continuación:
        \begin{minted}[
            frame=none,
            autogobble,
            obeytabs=false,
            breaklines,
            tabsize=4,
            linenos=true,
            % numbersep=-10pt,
            baselinestretch=1,
            firstnumber=1,
            bgcolor=bg!70,
            ]{prolog}
            %%% substractPeano/3 substracts S2 to S1, S3 = S1 - S2
            substractPeano(S1, S2, S3) :-
                peanoToNat(S1, N1),
                peanoToNat(S2, N2),
                N1 > N2,
                N3 is N1 - N2,
                natToPeano(N3, S3).
        \end{minted}
        Para ambas operaciones consideré oportuno hacer el cambio de ambos números de peano a su respectivo número natural, realizar la operación correspondiente y regresar a la representación de Peano.
    \end{solution}
    %
    \item Escriban un predicado \code{pino/1} cuyo argumento es un entero positivo y su salida es como sigue \textbf{(10 puntos)}:
    \begin{minted}[
        frame=none,
        autogobble,
        obeytabs=false,
        breaklines,
        tabsize=4,
        linenos=true,
        % numbersep=-10pt,
        baselinestretch=1,
        firstnumber=1,
        bgcolor=bg!70,
        ]{prolog}
        ?- pino(5).
            *
           * *
          * * *
         * * * *
        * * * * *
        true.
    \end{minted}
    \begin{solution}
        El programa para solucionar este inciso se muestra a continuación:
        \begin{minted}[
            frame=none,
            autogobble,
            obeytabs=false,
            breaklines,
            tabsize=4,
            linenos=true,
            % numbersep=-10pt,
            baselinestretch=1,
            firstnumber=1,
            bgcolor=bg!70,
            ]{prolog}
            % pine/1 creates a pine on a desired number of levels specified by Levels
            pine(Levels) :- pine(0, Levels), !.

            % pine/2 iterates through the lines appending spaces each line to center the stars
            pine(C, X) :- C < X, 
                C1 is C+1,
                Y1 is X - C1,
                spaces(Y1),
                stars(0, C),
                pine(C1, X).
            pine(C, X) :- C >= X.

            % spaces/1 writes N spaces on demand
            spaces(0) :- write('').
            spaces(N) :-
                N1 is N - 1,
                N1 >= 0,
                write(' '),
                spaces(N1).

            % stars/2 writes stars on demand on one line only and writes a new line after it has finished
            stars(X, Y) :- X =< Y,
                X1 is X + 1,
                write('* '),
                stars(X1, Y).
            stars(X, Y) :- X > Y, nl.
        \end{minted}
        Haciendo un query en Prolog para un pino de 7 pisos da el siguien resultado:
        \begin{minted}[
            frame=none,
            autogobble,
            obeytabs=false,
            breaklines,
            tabsize=4,
            linenos=true,
            % numbersep=-10pt,
            baselinestretch=1,
            firstnumber=1,
            bgcolor=bg!70,
            ]{prolog}
            ?- pine(7).
                  * 
                 * * 
                * * * 
               * * * * 
              * * * * * 
             * * * * * * 
            * * * * * * * 
            true.
        \end{minted}
        Jugando un poco con el programa se puede hacer un pino borracho:
        \begin{minted}[
            frame=none,
            autogobble,
            obeytabs=false,
            breaklines,
            tabsize=4,
            linenos=true,
            % numbersep=-10pt,
            baselinestretch=1,
            firstnumber=1,
            bgcolor=bg!70,
            ]{prolog}
            ?- pine(7).
                  *     
                 *     *     
                *     *     *     
               *     *     *     *     
              *     *     *     *     *     
             *     *     *     *     *     *     
            *     *     *     *     *     *     *     
            true.
        \end{minted}
    \end{solution}
    %
    \item Escriba un programa que regrese en su segundo argumento la lista de todas las permutaciones de la lista que es su primer argumento \textbf{(10 puntos)}. Por ejemplo:
    \begin{minted}[
        frame=none,
        autogobble,
        obeytabs=false,
        breaklines,
        tabsize=4,
        linenos=true,
        % numbersep=-10pt,
        baselinestretch=1,
        firstnumber=1,
        bgcolor=bg!70,
        ]{prolog}
        ?- perms([1,2,3], L).
        L = [[1, 2, 3], [1, 3, 2], [2, 1, 3], [2, 3, 1],
             [3, 1, 2], [3, 2, 1]].
    \end{minted}
    \begin{solution}
        Se hizo uso de \code{findall/3} y \code{permutation/2} definidas en la librería estándar de Prolog para realizar este ejercicio. Donde \code{findall/3} crea una lista de todas las instancias donde una meta definida tenga éxito, en este caso la meta es \code{permutation/2}.
        \begin{minted}[
            frame=none,
            autogobble,
            obeytabs=false,
            breaklines,
            tabsize=4,
            linenos=true,
            % numbersep=-10pt,
            baselinestretch=1,
            firstnumber=1,
            bgcolor=bg!70,
            ]{prolog}
            %%% perms/2 computes the all possible permutations of List in Perms.
            %%% Ejemplo:
            % ?- perms([1,2,3], L).
            % L = [[1, 2, 3], [1, 3, 2], [2, 1, 3], 
            %      [3, 1, 2], [2, 3, 1], [3, 2, 1]].
            perms(List, Perms) :-
                findall(Perm, permutation(Perm, List), Perms).
        \end{minted}
        Se mostraran un par de ejemplos de uso aparte del mostrado en el enunciado de la tarea el cual es incluido dentro de los comentarios de la solución:
        \begin{minted}[
            frame=none,
            autogobble,
            obeytabs=false,
            breaklines,
            tabsize=4,
            linenos=true,
            % numbersep=-10pt,
            baselinestretch=1,
            firstnumber=1,
            bgcolor=bg!70,
            ]{prolog}
            ?- perms([1,2], L); true.
            L = [[1, 2], [2, 1]] .
            ?- perms([1,2,3], L).
            L = [[1, 2, 3], [1, 3, 2], [2, 1, 3], [3, 1, 2], [2, 3, 1], [3, 2, 1]].
            ?- perms([1,2,3,4], L); true.
            L = [[1, 2, 3, 4], [1, 2, 4, 3], [1, 3, 2, 4], [1, 4, 2, 3], [1, 3, 4, 2], [1, 4, 3|...], [2, 1|...], [2|...], [...|...]|...] [write]
            L = [[1, 2, 3, 4], [1, 2, 4, 3], [1, 3, 2, 4], [1, 4, 2, 3], [1, 3, 4, 2], [1, 4, 3, 2], [2, 1, 3, 4], [2, 1, 4, 3], [3, 1, 2, 4], [4, 1, 2, 3], [3, 1, 4, 2], [4, 1, 3, 2], [2, 3, 1, 4], [2, 4, 1, 3], [3, 2, 1, 4], [4, 2, 1, 3], [3, 4, 1, 2], [4, 3, 1, 2], [2, 3, 4, 1], [2, 4, 3, 1], [3, 2, 4, 1], [4, 2, 3, 1], [3, 4, 2, 1], [4, 3, 2, 1]] .
        \end{minted}
        Para la última meta se especificó como \code{perms([1,2,3,4], L); true.} para poder tener la oportunidad de escribir \code{w} y mostrar la lista completa de permutaciones.
    \end{solution}
\end{enumerate}

% \vspace{1cm}
% \pagebreak

\bibliography{references.bib}

\end{document}
